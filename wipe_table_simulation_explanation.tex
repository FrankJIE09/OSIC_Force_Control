\documentclass[UTF8,12pt]{ctexart}
\usepackage{amsmath}
\usepackage{amssymb}
\usepackage{geometry}
\usepackage{listings}
\usepackage{xcolor}
\usepackage{hyperref}

\geometry{a4paper, margin=2.5cm}

\title{擦桌子仿真 - 混合力位控制算法说明}
\author{基于书本公式11.61实现}
\date{\today}

\begin{document}

\maketitle

\section{概述}

本文档描述擦桌子仿真中使用的混合运动-力控制算法。该算法基于书本公式11.61,实现了在约束环境下的混合控制策略。

\section{核心控制公式}

\subsection{完整公式(书本公式11.61)}

混合运动-力控制律的完整形式为:

\begin{equation}
\begin{split}
\tau = J_b^T(\theta)\Bigg[&P(\theta)\left(\tilde{\Lambda}(\theta)\frac{d}{dt}([Ad_{X^{-1}X_d}]V_d) + K_p X_e + K_i\int_0^t X_e(t)dt + K_d V_e\right) \\
&+ (I - P(\theta))\left(F_d + K_{fp}F_e + K_{fi}\int_0^t F_e(t)dt\right) + \tilde{\eta}(\theta, V_b)\Bigg],
\end{split}
\end{equation}

其中:
\begin{itemize}
    \item $\tau \in \mathbb{R}^n$:关节力矩向量
    \item $J_b(\theta) \in \mathbb{R}^{6 \times n}$:末端执行器坐标系$\{b\}$中的雅可比矩阵
    \item $P(\theta) \in \mathbb{R}^{6 \times 6}$:投影矩阵(公式11.63)
    \item $\tilde{\Lambda}(\theta) \in \mathbb{R}^{6 \times 6}$:任务空间质量矩阵
    \item $\tilde{\eta}(\theta, V_b) \in \mathbb{R}^6$:任务空间Coriolis项(末端执行器坐标系)
    \item $V_b \in \mathbb{R}^6$:末端执行器坐标系中的速度(twist)
    \item $X_e, V_e$:位置和速度误差
    \item $F_e$:力误差
\end{itemize}

\subsection{简化实现}

在实际代码实现中,对公式进行了以下简化:

\subsubsection{简化1:前馈加速度项}

\textbf{完整形式}:
\begin{equation}
\tilde{\Lambda}(\theta)\frac{d}{dt}([Ad_{X^{-1}X_d}]V_d)
\end{equation}

\textbf{简化说明}:
\begin{itemize}
    \item \textbf{Adjoint变换}:$[Ad_{X^{-1}X_d}]$ 表示从期望坐标系到当前坐标系的Adjoint变换
    \item \textbf{加速度项}:$\frac{d}{dt}([Ad_{X^{-1}X_d}]V_d)$ 需要计算期望速度的时间导数
    \item \textbf{简化处理}:对于平滑轨迹,此项通常很小,代码中简化为:
    \begin{equation}
    \tilde{\Lambda}(\theta) \cdot \mathbf{0} = \mathbf{0}
    \end{equation}
    即直接忽略前馈加速度项
\end{itemize}

\textbf{代码实现}(第872-879行):
\begin{lstlisting}[language=Python, basicstyle=\small]
# 计算前馈加速度项(简化实现:对于平滑轨迹,此项可忽略或很小)
V_dot_desired = self.compute_desired_acceleration_feedforward(
    pos_ref, self.quat_ref, vel_ref, pos_curr, quat_curr, V_b, dt
)

# 前馈项:Λ̃(θ) * V̇_d(简化:如果加速度项很小,可以忽略)
# 对于大多数平滑轨迹,此项可以忽略
feedforward_acceleration = Lambda @ V_dot_desired
\end{lstlisting}

\textbf{简化原因}:
\begin{enumerate}
    \item 对于平滑的擦拭轨迹,期望加速度$\dot{V}_d$通常很小
    \item Adjoint变换的计算需要额外的坐标变换,增加计算复杂度
    \item 在PID控制足够强的情况下,前馈项的影响较小
    \item 简化后仍能保证控制性能,同时提高实时性
\end{enumerate}

\subsubsection{简化2:Adjoint变换}

\textbf{完整形式}需要计算:
\begin{equation}
Ad_{X^{-1}X_d} = \begin{bmatrix}
R_{X^{-1}X_d} & 0 \\
[p]_{\times} R_{X^{-1}X_d} & R_{X^{-1}X_d}
\end{bmatrix}
\end{equation}

其中:
\begin{itemize}
    \item $X = (R, p)$:当前末端执行器位姿
    \item $X_d = (R_d, p_d)$:期望末端执行器位姿
    \item $R_{X^{-1}X_d} = R^{-1} R_d$:相对旋转矩阵
    \item $[p]_{\times}$:位置向量$p$的反对称矩阵
\end{itemize}

\textbf{简化处理}:代码中提供了\texttt{compute\_adjoint\_transformation}方法,但在实际控制中未使用,因为前馈加速度项已被忽略。

\section{实际实现的控制公式}

经过简化后,实际代码中使用的控制公式为:

\begin{equation}
\begin{split}
\tau = J_b^T(\theta)\Bigg[&P(\theta)\left(K_p X_e + K_i\int_0^t X_e(t)dt + K_d V_e\right) \\
&+ (I - P(\theta))\left(F_d + K_{fp}F_e + K_{fi}\int_0^t F_e(t)dt\right) + \tilde{\eta}(\theta, V_b)\Bigg]
\end{split}
\end{equation}

与完整公式相比,去掉了前馈加速度项:
\begin{equation}
\tilde{\Lambda}(\theta)\frac{d}{dt}([Ad_{X^{-1}X_d}]V_d) \approx \mathbf{0}
\end{equation}

\section{基础逻辑结构}

\subsection{控制流程}

算法的控制流程如下:

\begin{enumerate}
    \item \textbf{状态获取}:
    \begin{itemize}
        \item 获取当前关节角度$q$和角速度$\dot{q}$
        \item 计算雅可比矩阵$J_b(\theta)$
        \item 计算任务空间速度$V_b = J_b(\theta) \dot{q}$
        \item 获取当前末端位置$p_{curr}$和姿态$q_{curr}$
    \end{itemize}
    
    \item \textbf{动力学量计算}:
    \begin{itemize}
        \item 计算任务空间质量矩阵:$\tilde{\Lambda}(\theta) = (J_b M^{-1} J_b^T)^{-1}$
        \item 计算任务空间Coriolis项:$\tilde{\eta}(\theta, V_b)$
    \end{itemize}
    
    \item \textbf{约束和投影矩阵}:
    \begin{itemize}
        \item 根据接触状态确定约束矩阵$A(\theta)$(公式11.57)
        \item 计算投影矩阵$P(\theta)$(公式11.63)
    \end{itemize}
    
    \item \textbf{误差计算}:
    \begin{itemize}
        \item 位置误差:$X_e = p_{ref} - p_{curr}$
        \item 速度误差:$V_e = v_{ref} - V_b[:3]$
        \item 姿态误差:$e_{rot} = \text{quaternion\_error}(q_{ref}, q_{curr})$
        \item 力误差:$F_e = F_{desired} - F_{curr}$
    \end{itemize}
    
    \item \textbf{控制力计算}:
    \begin{itemize}
        \item 运动控制:$F_{motion} = P(\theta) \left(K_p X_e + K_i \int X_e + K_d V_e\right)$
        \item 力控制:$F_{force} = (I - P(\theta)) \left(F_d + K_{fp} F_e + K_{fi} \int F_e\right)$
        \item Coriolis补偿:$\tilde{\eta}(\theta, V_b)$(不投影)
        \item 组合:$F_{cmd} = F_{motion} + F_{force} + \tilde{\eta}(\theta, V_b)$
    \end{itemize}
    
    \item \textbf{关节力矩}:
    \begin{equation}
    \tau = J_b^T(\theta) F_{cmd} + g(\theta)
    \end{equation}
    其中$g(\theta)$是重力补偿项。
\end{enumerate}

\subsection{约束矩阵(公式11.57)}

对于表面接触任务,约束矩阵定义为:

\begin{equation}
A(\theta) = \begin{bmatrix}
0 & 0 & 1 & 0 & 0 & 0 \\  % 约束Z方向平移 v_z
0 & 0 & 0 & 1 & 0 & 0 \\  % 约束绕X轴旋转 ω_x
0 & 0 & 0 & 0 & 1 & 0 \\  % 约束绕Y轴旋转 ω_y
0 & 0 & 0 & 0 & 0 & 1    % 约束绕Z轴旋转 ω_z
\end{bmatrix}
\end{equation}

约束方程:
\begin{equation}
A(\theta) V_b = 0
\end{equation}

表示:
\begin{itemize}
    \item $v_z = 0$:Z方向平移速度为零(法向约束)
    \item $\omega_x = \omega_y = \omega_z = 0$:所有旋转速度为零(保持工具与表面平行)
\end{itemize}

\subsection{投影矩阵(公式11.63)}

投影矩阵将控制力分解到运动子空间和力子空间:

\begin{equation}
P(\theta) = I - A^T(\theta)(A(\theta)\tilde{\Lambda}^{-1}(\theta)A^T(\theta))^{-1}A(\theta)\tilde{\Lambda}^{-1}(\theta)
\end{equation}

性质:
\begin{itemize}
    \item $P(\theta)$:将力投影到运动子空间(rank = $6-k$,$k=4$为约束数量)
    \item $I - P(\theta)$:将力投影到力子空间(rank = $k = 4$)
    \item 正交性:$P(\theta) \cdot (I - P(\theta)) = 0$
\end{itemize}

\subsection{任务空间动力学}

\subsubsection{质量矩阵}

任务空间质量矩阵定义为:

\begin{equation}
\tilde{\Lambda}(\theta) = (J_b(\theta) M^{-1}(\theta) J_b^T(\theta))^{-1}
\end{equation}

其中$M(\theta) \in \mathbb{R}^{n \times n}$是关节空间质量矩阵。

\subsubsection{Coriolis项}

任务空间Coriolis项(公式8.91):

\begin{equation}
\tilde{\eta}(\theta, V_b) = J_b^{-T}(\theta) h(\theta, J_b^{-1}(\theta) V_b) - \tilde{\Lambda}(\theta) \dot{J}_b(\theta) J_b^{-1}(\theta) V_b
\end{equation}

其中:
\begin{itemize}
    \item $h(\theta, \dot{q})$:关节空间的科里奥利、向心力和重力项
    \item $\dot{J}_b(\theta)$:雅可比矩阵的时间导数
\end{itemize}

\textbf{重要}:根据书本公式11.61,Coriolis项$\tilde{\eta}(\theta, V_b)$不进行投影,直接添加到控制律中。

\section{控制增益}

\subsection{运动控制增益}

\begin{itemize}
    \item 位置增益:$K_p = \text{diag}(1000, 1000, 500)$
    \item 积分增益:$K_i = \text{diag}(20, 20, 10)$
    \item 微分增益:$K_d = \text{diag}(80, 80, 40)$
    \item 旋转增益:$K_{p,rot} = [50, 50, 30]$
\end{itemize}

\subsection{力控制增益}

\begin{itemize}
    \item 力比例增益:$K_{fp} = 1.0$
    \item 力积分增益:$K_{fi} = 0.2$
    \item 期望力:$F_{desired} = -15.0$ N(向下为负)
\end{itemize}

\section{简化总结}

与完整公式11.61相比,代码实现中进行了以下简化:

\begin{enumerate}
    \item \textbf{前馈加速度项}:$\tilde{\Lambda}(\theta)\frac{d}{dt}([Ad_{X^{-1}X_d}]V_d) \approx \mathbf{0}$
    \begin{itemize}
        \item 原因:对于平滑轨迹,此项影响很小
        \item 影响:略微降低跟踪精度,但提高实时性
        \item 适用场景:平滑的擦拭轨迹,PID控制足够强
    \end{itemize}
    
    \item \textbf{Adjoint变换}:未实现完整的$[Ad_{X^{-1}X_d}]$计算
    \begin{itemize}
        \item 原因:前馈项已忽略,无需计算
        \item 影响:无实际影响(因为前馈项为0)
    \end{itemize}
    
    \item \textbf{其他项}:完全按照书本公式实现
    \begin{itemize}
        \item 投影矩阵$P(\theta)$:完整实现
        \item Coriolis项$\tilde{\eta}(\theta, V_b)$:完整实现,不投影
        \item PID控制项:完整实现
        \item 力控制项:完整实现
    \end{itemize}
\end{enumerate}

\section{代码结构}

主要方法:

\begin{itemize}
    \item \texttt{control\_step()}:主控制循环,实现公式11.61
    \item \texttt{compute\_constraint\_matrix()}:计算约束矩阵$A(\theta)$
    \item \texttt{compute\_projection\_matrix()}:计算投影矩阵$P(\theta)$
    \item \texttt{compute\_desired\_acceleration\_feedforward()}:计算前馈加速度(简化实现,返回0)
    \item \texttt{compute\_adjoint\_transformation()}:计算Adjoint变换(已实现但未使用)
\end{itemize}

\section{总结}

代码实现了书本公式11.61的混合运动-力控制算法,主要简化了前馈加速度项。对于擦桌子这类平滑轨迹任务,该简化是合理的,能够在保证控制性能的同时提高实时性。其他核心部分(投影矩阵、Coriolis补偿、PID控制、力控制)均按照书本公式完整实现。

\end{document}

