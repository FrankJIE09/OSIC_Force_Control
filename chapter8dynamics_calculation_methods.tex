\documentclass{article}
\usepackage[UTF8]{ctex}
\usepackage{amsmath}
\usepackage{amssymb}
\usepackage{geometry}
\geometry{a4paper, margin=2.5cm}

\title{机器人动力学量的计算方法:$M(\theta)$、$h(\theta, \dot{\theta})$、$\Lambda(\theta)$和$\eta(\theta, V)$}
\author{}
\date{}

\begin{document}

\maketitle

\section{概述}

本文档详细说明机器人动力学中四个关键量的计算方法:
\begin{enumerate}
\item $M(\theta)$:关节空间质量矩阵
\item $h(\theta, \dot{\theta})$:科里奥利力、向心力和重力项
\item $\Lambda(\theta)$:任务空间质量矩阵
\item $\eta(\theta, V)$:任务空间科里奥利力、向心力和重力项
\end{enumerate}

\section{关节空间质量矩阵$M(\theta)$的计算}

\subsection{基本定义}

质量矩阵$M(\theta) \in \mathbb{R}^{n \times n}$是一个对称正定矩阵,描述了关节加速度$\ddot{\theta}$与关节力矩$\tau$之间的关系。在动力学方程中:
\begin{equation}
\tau = M(\theta)\ddot{\theta} + h(\theta, \dot{\theta})
\end{equation}

\subsection{计算方法1:拉格朗日方法}

\subsubsection{基本原理}

拉格朗日方法通过系统的动能和势能来推导动力学方程。

\textbf{步骤}:

\begin{enumerate}
\item \textbf{计算系统动能}:
\begin{equation}
T = \frac{1}{2}\sum_{i=1}^{n}\sum_{j=1}^{n} M_{ij}(\theta)\dot{\theta}_i\dot{\theta}_j = \frac{1}{2}\dot{\theta}^T M(\theta)\dot{\theta}
\end{equation}

\item \textbf{从动能提取质量矩阵}:
质量矩阵的元素可以通过动能的二阶偏导数得到:
\begin{equation}
M_{ij}(\theta) = \frac{\partial^2 T}{\partial \dot{\theta}_i \partial \dot{\theta}_j}
\end{equation}

\item \textbf{具体计算过程}:
\begin{itemize}
\item 对于每个连杆$i$,计算其动能:
\begin{equation}
T_i = \frac{1}{2}m_i v_{c,i}^T v_{c,i} + \frac{1}{2}\omega_i^T I_i \omega_i
\end{equation}
其中:
\begin{itemize}
\item $m_i$:连杆$i$的质量
\item $v_{c,i}$:连杆$i$质心的线速度
\item $\omega_i$:连杆$i$的角速度
\item $I_i$:连杆$i$在质心处的惯性张量
\end{itemize}

\item 总动能为所有连杆动能之和:
\begin{equation}
T = \sum_{i=1}^{n} T_i
\end{equation}

\item 将速度$v_{c,i}$和$\omega_i$表示为关节速度$\dot{\theta}$的函数:
\begin{align}
v_{c,i} &= J_{v,i}(\theta)\dot{\theta} \\
\omega_i &= J_{\omega,i}(\theta)\dot{\theta}
\end{align}
其中$J_{v,i}$和$J_{\omega,i}$分别是连杆$i$质心的线速度和角速度雅可比矩阵。

\item 代入动能表达式:
\begin{align}
T_i &= \frac{1}{2}m_i (J_{v,i}\dot{\theta})^T (J_{v,i}\dot{\theta}) + \frac{1}{2}(J_{\omega,i}\dot{\theta})^T I_i (J_{\omega,i}\dot{\theta}) \\
    &= \frac{1}{2}\dot{\theta}^T [m_i J_{v,i}^T J_{v,i} + J_{\omega,i}^T I_i J_{\omega,i}]\dot{\theta}
\end{align}

\item 质量矩阵为:
\begin{equation}
M(\theta) = \sum_{i=1}^{n} [m_i J_{v,i}^T(\theta) J_{v,i}(\theta) + J_{\omega,i}^T(\theta) I_i J_{\omega,i}(\theta)]
\end{equation}
\end{itemize}
\end{enumerate}

\subsubsection{计算复杂度}

\begin{itemize}
\item 需要计算每个连杆的雅可比矩阵$J_{v,i}$和$J_{\omega,i}$
\item 对于$n$关节机器人,需要计算$n$个$3 \times n$矩阵(线速度)和$n$个$3 \times n$矩阵(角速度)
\item 总计算复杂度:$O(n^3)$
\end{itemize}

\subsection{计算方法2:递归牛顿-欧拉方法(推荐)}

递归牛顿-欧拉方法是一种更高效的计算方法,特别适合实时控制。

\subsubsection{前向递推(Forward Recursion)}

从基座到末端,计算每个连杆的速度和加速度:

\begin{enumerate}
\item \textbf{初始化}(基座):
\begin{align}
\omega_0 &= 0 \\
\dot{\omega}_0 &= 0 \\
v_0 &= 0 \\
\dot{v}_0 &= g \quad \text{(重力加速度)}
\end{align}

\item \textbf{对于每个关节$i = 1, \ldots, n$}:
\begin{itemize}
\item 如果关节$i$是旋转关节:
\begin{align}
\omega_i &= R_i^T \omega_{i-1} + \dot{\theta}_i z_i \\
\dot{\omega}_i &= R_i^T \dot{\omega}_{i-1} + R_i^T \omega_{i-1} \times \dot{\theta}_i z_i + \ddot{\theta}_i z_i \\
v_i &= R_i^T (v_{i-1} + \omega_{i-1} \times p_i) \\
\dot{v}_i &= R_i^T (\dot{v}_{i-1} + \dot{\omega}_{i-1} \times p_i + \omega_{i-1} \times (\omega_{i-1} \times p_i))
\end{align}

\item 如果关节$i$是移动关节:
\begin{align}
\omega_i &= R_i^T \omega_{i-1} \\
\dot{\omega}_i &= R_i^T \dot{\omega}_{i-1} \\
v_i &= R_i^T (v_{i-1} + \omega_{i-1} \times p_i) + \dot{d}_i z_i \\
\dot{v}_i &= R_i^T (\dot{v}_{i-1} + \dot{\omega}_{i-1} \times p_i + \omega_{i-1} \times (\omega_{i-1} \times p_i)) + 2\omega_i \times \dot{d}_i z_i + \ddot{d}_i z_i
\end{align}

\item 计算质心加速度:
\begin{align}
\dot{v}_{c,i} &= \dot{v}_i + \dot{\omega}_i \times r_i + \omega_i \times (\omega_i \times r_i)
\end{align}
其中$r_i$是从关节$i$到连杆$i$质心的向量。
\end{itemize}
\end{enumerate}

\subsubsection{后向递推(Backward Recursion)}

从末端到基座,计算每个连杆的力和力矩:

\begin{enumerate}
\item \textbf{初始化}(末端):
\begin{equation}
f_{n+1} = 0, \quad \mu_{n+1} = 0
\end{equation}

\item \textbf{对于每个关节$i = n, n-1, \ldots, 1$}:
\begin{align}
f_i &= R_{i+1} f_{i+1} + m_i \dot{v}_{c,i} \\
\mu_i &= R_{i+1} \mu_{i+1} + r_i \times f_i + I_i \dot{\omega}_i + \omega_i \times (I_i \omega_i)
\end{align}

\item \textbf{计算关节力矩}:
\begin{itemize}
\item 如果关节$i$是旋转关节:
\begin{equation}
\tau_i = \mu_i^T z_i
\end{equation}

\item 如果关节$i$是移动关节:
\begin{equation}
\tau_i = f_i^T z_i
\end{equation}
\end{itemize}
\end{enumerate}

\subsubsection{质量矩阵的计算}

为了计算质量矩阵$M(\theta)$,需要对每个关节$j$,设置$\ddot{\theta}_j = 1$,其他$\ddot{\theta}_k = 0$($k \neq j$),并设置所有$\dot{\theta}_k = 0$和重力$g = 0$,然后运行递归算法。得到的$\tau_i$就是$M_{ij}(\theta)$。

\textbf{算法}:
\begin{enumerate}
\item 对于$j = 1, \ldots, n$:
\begin{itemize}
\item 设置$\ddot{\theta}_j = 1$,$\ddot{\theta}_k = 0$($k \neq j$)
\item 设置所有$\dot{\theta}_k = 0$
\item 设置$g = 0$(忽略重力)
\item 运行递归牛顿-欧拉算法
\item $M_{ij}(\theta) = \tau_i$(对于所有$i$)
\end{itemize}
\end{enumerate}

\subsubsection{计算复杂度}

\begin{itemize}
\item 对于每个$j$,递归算法需要$O(n)$时间
\item 总共需要$n$次递归,总复杂度:$O(n^2)$
\item 比拉格朗日方法更高效
\end{itemize}

\section{科里奥利力、向心力和重力项$h(\theta, \dot{\theta})$的计算}

\subsection{基本定义}

$h(\theta, \dot{\theta})$包含三项:
\begin{equation}
h(\theta, \dot{\theta}) = C(\theta, \dot{\theta})\dot{\theta} + g(\theta) + b(\dot{\theta})
\end{equation}
其中:
\begin{itemize}
\item $C(\theta, \dot{\theta})\dot{\theta}$:科里奥利力和向心力项
\item $g(\theta)$:重力项
\item $b(\dot{\theta})$:摩擦项(可选)
\end{itemize}

\subsection{计算方法1:从拉格朗日方程}

从拉格朗日方程:
\begin{equation}
\frac{d}{dt}\frac{\partial L}{\partial \dot{\theta}} - \frac{\partial L}{\partial \theta} = \tau
\end{equation}
其中$L = T - U$是拉格朗日量,$T$是动能,$U$是势能。

展开后得到:
\begin{equation}
M(\theta)\ddot{\theta} + \dot{M}(\theta)\dot{\theta} - \frac{1}{2}\frac{\partial}{\partial \theta}[\dot{\theta}^T M(\theta)\dot{\theta}] + \frac{\partial U}{\partial \theta} = \tau
\end{equation}

因此:
\begin{align}
C(\theta, \dot{\theta})\dot{\theta} &= \dot{M}(\theta)\dot{\theta} - \frac{1}{2}\frac{\partial}{\partial \theta}[\dot{\theta}^T M(\theta)\dot{\theta}] \\
g(\theta) &= \frac{\partial U}{\partial \theta}
\end{align}

\subsection{计算方法2:递归牛顿-欧拉方法(推荐)}

\subsubsection{重力项$g(\theta)$的计算}

\begin{enumerate}
\item 设置所有$\dot{\theta}_i = 0$和$\ddot{\theta}_i = 0$
\item 运行递归牛顿-欧拉算法(包含重力$g$)
\item 得到的$\tau_i$就是$g_i(\theta)$
\end{enumerate}

\subsubsection{科里奥利项$C(\theta, \dot{\theta})\dot{\theta}$的计算}

\begin{enumerate}
\item 设置重力$g = 0$
\item 使用给定的$\theta$和$\dot{\theta}$运行递归牛顿-欧拉算法
\item 计算$\tau_{\text{total}} = M(\theta)\ddot{\theta} + C(\theta, \dot{\theta})\dot{\theta} + g(\theta)$
\item 由于$\ddot{\theta} = 0$(在计算$h$时),且$g(\theta)$已单独计算,因此:
\begin{equation}
C(\theta, \dot{\theta})\dot{\theta} = \tau_{\text{total}} - g(\theta)
\end{equation}
\end{enumerate}

\subsubsection{完整算法}

\textbf{计算$h(\theta, \dot{\theta})$}:
\begin{enumerate}
\item \textbf{计算重力项}:
\begin{itemize}
\item 设置$\dot{\theta} = 0$,$\ddot{\theta} = 0$
\item 运行递归算法(包含重力)
\item $g(\theta) = \tau$
\end{itemize}

\item \textbf{计算科里奥利项}:
\begin{itemize}
\item 设置$\ddot{\theta} = 0$,$g = 0$
\item 使用给定的$\dot{\theta}$运行递归算法
\item $C(\theta, \dot{\theta})\dot{\theta} = \tau - g(\theta)$
\end{itemize}

\item \textbf{组合}:
\begin{equation}
h(\theta, \dot{\theta}) = C(\theta, \dot{\theta})\dot{\theta} + g(\theta)
\end{equation}
\end{enumerate}

\section{任务空间质量矩阵$\Lambda(\theta)$的计算}

\subsection{基本定义}

任务空间质量矩阵定义为:
\begin{equation}
\Lambda(\theta) = J^{-T}(\theta)M(\theta)J^{-1}(\theta) \tag{8.90}
\end{equation}
其中$J(\theta)$是雅可比矩阵(可以是空间雅可比$J_s$或体雅可比$J_b$)。

\subsection{计算方法}

\subsubsection{方法1:直接计算(当$J$可逆时)}

\begin{enumerate}
\item 计算关节空间质量矩阵$M(\theta)$(使用前面介绍的方法)
\item 计算雅可比矩阵$J(\theta)$
\item 计算$J^{-1}(\theta)$(如果$J$可逆)
\item 计算:
\begin{equation}
\Lambda(\theta) = J^{-T}(\theta)M(\theta)J^{-1}(\theta)
\end{equation}
\end{enumerate}

\textbf{注意}:
\begin{itemize}
\item 需要$J(\theta)$可逆,即机器人不在奇异点
\item 对于6自由度机器人,$\Lambda(\theta)$是$6 \times 6$矩阵
\item 计算复杂度:$O(n^3)$(主要是矩阵求逆和乘法)
\end{itemize}

\subsubsection{方法2:使用伪逆(当$J$不可逆或冗余机器人时)}

对于冗余机器人($n > 6$)或接近奇异点的情况,使用伪逆:
\begin{equation}
\Lambda(\theta) = (J^{\dagger})^T M(\theta) J^{\dagger}
\end{equation}
其中$J^{\dagger} = (J^T J)^{-1}J^T$是伪逆。

\subsubsection{方法3:避免求逆(数值稳定)}

为了避免直接求逆,可以使用:
\begin{equation}
\Lambda(\theta) = (J M^{-1} J^T)^{-1}
\end{equation}

\textbf{验证}:
\begin{align}
(J M^{-1} J^T)^{-1} &= (J^T)^{-1} M (J^{-1}) \\
                    &= J^{-T} M J^{-1} = \Lambda(\theta)
\end{align}

\textbf{优点}:
\begin{itemize}
\item 只需要计算$M^{-1}$(如果$M$是对角或稀疏的,这很容易)
\item 数值上更稳定
\end{itemize}

\section{任务空间科里奥利项$\eta(\theta, V)$的计算}

\subsection{基本定义}

任务空间科里奥利项定义为:
\begin{equation}
\eta(\theta, V) = J^{-T}(\theta)h(\theta, J^{-1}V) - \Lambda(\theta)\dot{J}(\theta)J^{-1}(\theta)V \tag{8.91}
\end{equation}

\subsection{计算方法}

\subsubsection{步骤1:计算关节速度}

从任务空间速度$V$计算关节速度:
\begin{equation}
\dot{\theta} = J^{-1}(\theta)V
\end{equation}

\subsubsection{步骤2:计算$h(\theta, \dot{\theta})$}

使用前面介绍的方法计算:
\begin{equation}
h(\theta, \dot{\theta}) = h(\theta, J^{-1}V)
\end{equation}

\subsubsection{步骤3:计算雅可比矩阵的时间导数$\dot{J}(\theta)$}

雅可比矩阵的时间导数为:
\begin{equation}
\dot{J}(\theta) = \frac{d}{dt}J(\theta) = \sum_{i=1}^{n}\frac{\partial J}{\partial \theta_i}\dot{\theta}_i
\end{equation}

\textbf{计算方法}:
\begin{itemize}
\item 对于每个关节$i$,计算偏导数$\frac{\partial J}{\partial \theta_i}$
\item 使用链式法则:
\begin{equation}
\dot{J}(\theta) = \sum_{i=1}^{n}\frac{\partial J}{\partial \theta_i}\dot{\theta}_i
\end{equation}
\end{itemize}

\textbf{数值方法}(如果解析计算困难):
\begin{equation}
\frac{\partial J}{\partial \theta_i} \approx \frac{J(\theta + \epsilon e_i) - J(\theta)}{\epsilon}
\end{equation}
其中$e_i$是第$i$个单位向量,$\epsilon$是小量。

\subsubsection{步骤4:计算$\Lambda(\theta)$}

使用前面介绍的方法计算任务空间质量矩阵$\Lambda(\theta)$。

\subsubsection{步骤5:组合各项}

\begin{equation}
\eta(\theta, V) = J^{-T}(\theta)h(\theta, J^{-1}V) - \Lambda(\theta)\dot{J}(\theta)J^{-1}(\theta)V
\end{equation}

\subsection{完整算法}

\textbf{计算$\eta(\theta, V)$}:
\begin{enumerate}
\item 计算$\dot{\theta} = J^{-1}(\theta)V$
\item 计算$h(\theta, \dot{\theta})$(使用递归牛顿-欧拉方法)
\item 计算$\dot{J}(\theta) = \sum_{i=1}^{n}\frac{\partial J}{\partial \theta_i}\dot{\theta}_i$
\item 计算$\Lambda(\theta) = J^{-T}(\theta)M(\theta)J^{-1}(\theta)$
\item 计算第一项:$J^{-T}(\theta)h(\theta, \dot{\theta})$
\item 计算第二项:$\Lambda(\theta)\dot{J}(\theta)J^{-1}(\theta)V$
\item 组合:$\eta(\theta, V) = \text{第一项} - \text{第二项}$
\end{enumerate}

\section{计算复杂度总结}

\begin{table}[h]
\centering
\begin{tabular}{|l|l|l|}
\hline
\textbf{量} & \textbf{方法} & \textbf{复杂度} \\
\hline
$M(\theta)$ & 拉格朗日方法 & $O(n^3)$ \\
$M(\theta)$ & 递归牛顿-欧拉 & $O(n^2)$ \\
$h(\theta, \dot{\theta})$ & 递归牛顿-欧拉 & $O(n)$ \\
$\Lambda(\theta)$ & 直接计算 & $O(n^3)$ \\
$\eta(\theta, V)$ & 完整计算 & $O(n^3)$ \\
\hline
\end{tabular}
\caption{计算复杂度总结}
\end{table}

\section{实际实现建议}

\subsection{使用递归牛顿-欧拉方法}

\begin{itemize}
\item \textbf{推荐}:使用递归牛顿-欧拉方法计算$M(\theta)$和$h(\theta, \dot{\theta})$
\item \textbf{优点}:
\begin{itemize}
\item 计算效率高($O(n^2)$和$O(n)$)
\item 数值稳定
\item 适合实时控制
\end{itemize}
\end{itemize}

\subsection{缓存中间结果}

\begin{itemize}
\item 雅可比矩阵$J(\theta)$:如果$\theta$没有变化,可以重用
\item 质量矩阵$M(\theta)$:如果$\theta$没有变化,可以重用
\item $\Lambda(\theta)$:依赖于$M(\theta)$和$J(\theta)$,如果两者都没变化,可以重用
\end{itemize}

\subsection{处理奇异点}

\begin{itemize}
\item 检测雅可比矩阵的条件数
\item 在接近奇异点时,使用阻尼伪逆:
\begin{equation}
J^{\dagger} = (J^T J + \lambda I)^{-1}J^T
\end{equation}
其中$\lambda$是小的正数
\end{itemize}

\section{总结}

\begin{enumerate}
\item \textbf{$M(\theta)$}:使用递归牛顿-欧拉方法,设置$\ddot{\theta}_j = 1$,其他为0,计算$n$次得到所有列

\item \textbf{$h(\theta, \dot{\theta})$}:
\begin{itemize}
\item 重力项:设置$\dot{\theta} = 0$,$\ddot{\theta} = 0$,运行递归算法
\item 科里奥利项:设置$\ddot{\theta} = 0$,$g = 0$,使用给定$\dot{\theta}$运行递归算法
\end{itemize}

\item \textbf{$\Lambda(\theta)$}:$\Lambda(\theta) = J^{-T}(\theta)M(\theta)J^{-1}(\theta)$

\item \textbf{$\eta(\theta, V)$}:$\eta(\theta, V) = J^{-T}(\theta)h(\theta, J^{-1}V) - \Lambda(\theta)\dot{J}(\theta)J^{-1}(\theta)V$
\end{enumerate}

这些计算方法为机器人动力学分析和控制提供了基础。

\end{document}