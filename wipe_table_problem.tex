\documentclass[12pt,a4paper]{article}
\usepackage[UTF8]{ctex}
\usepackage{amsmath}
\usepackage{amssymb}
\usepackage{geometry}
\geometry{margin=2.5cm}

\title{擦桌子任务:基于混合力位控制的机器人表面操作}
\author{题目}
\date{}

\begin{document}

\maketitle

\section{问题描述}

考虑一个7自由度Franka Panda机械臂执行擦桌子任务。机器人末端执行器需要:
\begin{enumerate}
    \item 在水平工作表面上进行擦拭运动(XY平面内的切向运动)
    \item 施加恒定的法向力(Z方向,目标力$F_d = -15$ N,向下为负)
    \item 保持工具姿态与表面平行(Z轴垂直于表面向下)
\end{enumerate}

\section{任务要求}

设计一个基于约束的混合运动-力控制器,实现以下功能:

\subsection{约束条件}
当末端执行器与表面接触时,需要满足以下约束:
\begin{itemize}
    \item \textbf{法向平移约束}:$v_z = 0$(末端执行器不能沿Z方向移动)
    \item \textbf{旋转约束}:$\omega_x = \omega_y = \omega_z = 0$(保持工具与表面平行,不允许旋转)
\end{itemize}

\subsection{控制目标}
\begin{itemize}
    \item \textbf{运动子空间}:在XY平面内跟踪期望的擦拭轨迹(位置和速度控制)
    \item \textbf{力子空间}:在Z方向施加期望的法向力(力控制)
    \item \textbf{姿态控制}:通过旋转约束保持工具姿态稳定
\end{itemize}

\subsection{系统参数}
\begin{itemize}
    \item 机器人:7自由度Franka Panda机械臂
    \item 工作表面:水平平面,位于$z = 0.15$ m
    \item 期望法向力:$F_d = -15$ N(向下)
    \item 末端执行器坐标系:$\{b\}$(body frame)
\end{itemize}

\section{需要推导的公式}

请基于混合运动-力控制理论(公式11.61),推导并写出以下内容:

\begin{enumerate}
    \item 约束矩阵 $A(\theta)$(公式11.57)
    \item 投影矩阵 $P(\theta)$(公式11.63)
    \item 配置误差 $X_e$(基于SE(3)的log映射)
    \item 速度误差 $V_e$(考虑Adjoint变换)
    \item 完整的控制律 $\tau$(公式11.61)
    \item 任务空间质量矩阵 $\tilde{\Lambda}(\theta)$
    \item 任务空间Coriolis项 $\tilde{\eta}(\theta, V_b)$
\end{enumerate}

\section{控制律结构}

控制律应具有以下形式(公式11.61):
\begin{equation}
\tau = J_b^T(\theta)\left[P(\theta)F_{\text{motion}} + (I-P(\theta))F_{\text{force}} + \tilde{\eta}(\theta, V_b)\right]
\end{equation}

其中:
\begin{itemize}
    \item $F_{\text{motion}}$:运动控制wrench(投影到运动子空间)
    \item $F_{\text{force}}$:力控制wrench(投影到力子空间)
    \item $P(\theta)$:投影矩阵,将wrench投影到运动子空间
    \item $I-P(\theta)$:投影矩阵,将wrench投影到力子空间
\end{itemize}

\end{document}

