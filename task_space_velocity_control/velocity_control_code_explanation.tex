\documentclass[12pt,a4paper]{article}
\usepackage[UTF8]{ctex}
\usepackage{amsmath,amssymb,amsthm}
\usepackage{graphicx}
\usepackage{geometry}
\usepackage{hyperref}
\usepackage{algorithm}
\usepackage{algorithmic}
\usepackage{listings}
\usepackage{xcolor}

\geometry{margin=2.5cm}

% 代码样式设置
\lstset{
    language=Python,
    basicstyle=\ttfamily\small,
    keywordstyle=\color{blue}\bfseries,
    commentstyle=\color{green!60!black},
    stringstyle=\color{red},
    numbers=left,
    numberstyle=\tiny\color{gray},
    stepnumber=1,
    numbersep=5pt,
    backgroundcolor=\color{gray!10},
    frame=single,
    breaklines=true,
    breakatwhitespace=true,
    tabsize=4,
    showspaces=false,
    showstringspaces=false,
    showtabs=false
}

\title{任务空间速度控制代码说明\\
\large 基于公式236的6DOF扩展实现}
\author{OSIC Force Control Project}
\date{\today}

\begin{document}

\maketitle

\section{概述}

本文档介绍基于公式236实现的任务空间速度控制代码。该代码使用MuJoCo仿真器实现Franka Panda机械臂的任务空间速度控制,控制律为:

\begin{equation}
\dot{\theta} = J^{-1}(\theta)\left[\dot{X}_d + K_p X_e\right]
\end{equation}

其中:
\begin{itemize}
    \item $\dot{\theta}$:关节速度命令(7维)
    \item $J(\theta)$:雅可比矩阵(6×7)
    \item $\dot{X}_d$:参考末端执行器速度(6维twist)
    \item $K_p$:比例增益矩阵(6×6)
    \item $X_e = \log(X^{-1}X_d)$:配置误差(6维)
\end{itemize}

\section{控制律的6DOF扩展}

公式236原本是针对位置控制(3DOF)的,本代码将其扩展到6DOF(位置+姿态):

\begin{equation}
\dot{\theta} = J^+(\theta)\left[\dot{X}_d + K_p X_e\right]
\end{equation}

其中:
\begin{itemize}
    \item $X_e = \log(X^{-1}X_d)$ 使用SE(3)的log映射计算6维配置误差
    \item $\dot{X}_d = [\omega_d, v_d]^T$ 是6维twist(角速度在前,线速度在后)
    \item $J^+$ 是雅可比矩阵的伪逆,用于处理冗余机器人(7自由度)
\end{itemize}

\section{代码结构}

\subsection{主要类:TaskSpaceVelocitySimulation}

代码的核心是\texttt{TaskSpaceVelocitySimulation}类,主要包含以下方法:

\subsubsection{初始化方法}

\begin{lstlisting}[caption=初始化方法]
def __init__(self, model_path: str = "surface_force_control.xml"):
    # 加载MuJoCo模型
    self.model = mujoco.MjModel.from_xml_path(model_path)
    self.data = mujoco.MjData(self.model)
    
    # 获取关节信息
    self.joint_names = [f"panda_joint{i + 1}" for i in range(7)]
    self.velocity_actuator_names = [f"vel_panda_joint{i + 1}" for i in range(7)]
    
    # 设置初始配置
    q_init = self.compute_initial_configuration()
    
    # 设置控制参数
    self.setup_control_parameters()
\end{lstlisting}

\textbf{关键点:}
\begin{itemize}
    \item 使用速度执行器(velocity actuators)而不是力矩执行器(motor)
    \item 执行器名称格式:\texttt{vel\_panda\_joint1} 到 \texttt{vel\_panda\_joint7}
    \item 不需要动力学计算器(速度控制不需要动力学模型)
\end{itemize}

\subsubsection{控制参数设置}

\begin{lstlisting}[caption=控制参数设置]
def setup_control_parameters(self):
    """设置控制参数(公式236)"""
    # 6维任务空间控制增益(角速度在前)
    # K_p: 位置和姿态比例增益
    self.K_p = np.diag([5.0, 5.0, 5.0, 3.0, 3.0, 3.0])  # [旋转, 位置]
\end{lstlisting}

\textbf{说明:}
\begin{itemize}
    \item 仅使用比例增益$K_p$,速度控制通常不需要积分和微分项
    \item 前3个元素对应旋转(姿态)控制,后3个元素对应位置控制
    \item 旋转增益通常设置得比位置增益稍大
\end{itemize}

\subsubsection{雅可比矩阵计算}

\begin{lstlisting}[caption=雅可比矩阵计算]
def get_jacobian_6x7(self):
    """获取6DOF雅可比矩阵(位置+姿态)"""
    jacp = np.zeros((3, self.model.nv))  # 位置雅可比
    jacr = np.zeros((3, self.model.nv))  # 旋转雅可比
    
    hand_body_id = self.model.body("panda_hand").id
    mujoco.mj_jacBody(self.model, self.data, jacp, jacr, hand_body_id)
    
    J = np.zeros((6, self.n_joints))
    for i, name in enumerate(self.joint_names):
        dof_idx = self.joint_dof_indices[i]
        if dof_idx >= 0:
            J[:3, i] = jacr[:, dof_idx]  # 角速度部分
            J[3:, i] = jacp[:, dof_idx]   # 线速度部分
    
    return J
\end{lstlisting}

\textbf{重要约定:}
\begin{itemize}
    \item 使用书本第513行约定:角速度在前,线速度在后
    \item $J[:3, :]$ 对应角速度部分($\omega_x, \omega_y, \omega_z$)
    \item $J[3:, :]$ 对应线速度部分($v_x, v_y, v_z$)
\end{itemize}

\subsubsection{配置误差计算}

\begin{lstlisting}[caption=配置误差计算(SE(3) log映射)]
def compute_configuration_error(self, pos_curr, quat_curr, pos_ref, quat_ref):
    """计算配置误差 X_e = log(X^{-1} X_d)"""
    # 转换为旋转矩阵
    rot_curr = Rotation.from_quat(q_curr_scipy)
    rot_ref = Rotation.from_quat(q_ref_scipy)
    R_curr = rot_curr.as_matrix()
    R_ref = rot_ref.as_matrix()
    
    # 构建SE(3)变换矩阵
    T_curr = np.eye(4)
    T_curr[:3, :3] = R_curr
    T_curr[:3, 3] = pos_curr
    
    T_ref = np.eye(4)
    T_ref[:3, :3] = R_ref
    T_ref[:3, 3] = pos_ref
    
    # 计算相对变换:T_err = T_curr^{-1} @ T_ref
    T_curr_inv = np.eye(4)
    T_curr_inv[:3, :3] = R_curr.T
    T_curr_inv[:3, 3] = -R_curr.T @ pos_curr
    T_err = T_curr_inv @ T_ref
    
    # 计算log映射:X_e = log(T_err)
    X_e = self.se3_log_map(T_err[:3, :3], T_err[:3, 3])
    return X_e
\end{lstlisting}

\textbf{说明:}
\begin{itemize}
    \item 使用SE(3)的log映射计算6维配置误差
    \item 误差在末端执行器坐标系中表示
    \item 前3维是旋转误差(旋转向量),后3维是位置误差
\end{itemize}

\subsubsection{控制律实现}

\begin{lstlisting}[caption=控制律实现(核心代码)]
def control_step(self, t, dt=0.002):
    """执行一步任务空间速度控制 - 公式236(6DOF扩展)"""
    
    # 1. 获取当前状态
    pos_curr = np.array(self.data.body("panda_hand").xpos)
    quat_curr = np.array(self.data.body("panda_hand").xquat)
    
    # 2. 计算雅可比矩阵
    J = self.get_jacobian_6x7()
    
    # 3. 生成参考轨迹
    z_des = 0.4 - min(t / 2.0, 1.0) * 0.25
    pos_ref = np.array([0.4, 0.0, z_des])
    vel_ref_linear = np.array([0.0, 0.0, -0.25 / 2.0 if t < 2.0 else 0.0])
    vel_ref_angular = np.zeros(3)
    vel_ref_6d = np.concatenate([vel_ref_angular, vel_ref_linear])
    
    # 4. 计算配置误差
    X_e = self.compute_configuration_error(
        pos_curr, quat_curr, pos_ref, quat_ref
    )
    
    # 5. 任务空间速度控制律(公式236)
    # θ̇ = J^{-1}(θ)[Ẋ_d + K_p X_e]
    X_dot_desired = vel_ref_6d + self.K_p @ X_e
    
    # 6. 使用伪逆计算关节速度
    J_pinv = np.linalg.pinv(J, rcond=1e-6)
    qdot_cmd = J_pinv @ X_dot_desired
    
    # 7. 速度限制
    qdot_max = np.array([2.0, 2.0, 2.0, 2.0, 2.5, 2.5, 2.5])
    qdot_cmd = np.clip(qdot_cmd, -qdot_max, qdot_max)
    
    # 8. 应用控制(速度控制)
    for i, name in enumerate(self.velocity_actuator_names):
        ctrl_id = self.model.actuator(name).id
        self.data.actuator(ctrl_id).ctrl = float(qdot_cmd[i])
\end{lstlisting}

\textbf{控制流程:}
\begin{enumerate}
    \item 获取当前末端执行器位姿(位置和姿态)
    \item 计算6×7雅可比矩阵
    \item 生成参考轨迹(位置和速度)
    \item 计算6维配置误差$X_e = \log(X^{-1}X_d)$
    \item 计算期望任务空间速度:$\dot{X}_d + K_p X_e$
    \item 使用雅可比伪逆计算关节速度:$\dot{\theta} = J^+ \dot{X}_{desired}$
    \item 限制关节速度在安全范围内
    \item 将速度命令发送给速度执行器
\end{enumerate}

\section{与力矩控制版本的区别}

\subsection{主要差异}

\begin{table}[h]
\centering
\begin{tabular}{|l|l|l|}
\hline
\textbf{特性} & \textbf{速度控制(公式236)} & \textbf{力矩控制(公式11.37)} \\
\hline
控制律 & $\dot{\theta} = J^{-1}[\dot{X}_d + K_p X_e]$ & $\tau = J^T[\Lambda \ddot{X}_d + K_p X_e + K_d V_e] + \eta$ \\
\hline
执行器类型 & velocity & motor \\
\hline
输出量 & 关节速度 $\dot{\theta}$ & 关节力矩 $\tau$ \\
\hline
控制参数 & 仅$K_p$ & $K_p, K_i, K_d$ \\
\hline
动力学模型 & 不需要 & 需要(质量矩阵、Coriolis项) \\
\hline
雅可比运算 & 伪逆 $J^+$ & 转置 $J^T$ \\
\hline
前馈项 & 参考速度 $\dot{X}_d$ & 参考加速度 $\ddot{X}_d$ \\
\hline
\end{tabular}
\caption{速度控制与力矩控制的主要区别}
\end{table}

\subsection{代码层面的差异}

\begin{enumerate}
    \item \textbf{执行器设置}:
    \begin{itemize}
        \item 速度控制:使用\texttt{<velocity>}执行器,名称格式为\texttt{vel\_panda\_joint*}
        \item 力矩控制:使用\texttt{<motor>}执行器,名称格式为\texttt{panda\_joint*}
    \end{itemize}
    
    \item \textbf{控制参数}:
    \begin{itemize}
        \item 速度控制:仅需设置$K_p$比例增益
        \item 力矩控制:需要$K_p, K_i, K_d$三个增益矩阵
    \end{itemize}
    
    \item \textbf{动力学计算}:
    \begin{itemize}
        \item 速度控制:不需要计算质量矩阵和Coriolis项
        \item 力矩控制:需要计算$\Lambda(\theta)$和$\eta(\theta, V_b)$
    \end{itemize}
    
    \item \textbf{控制输出}:
    \begin{itemize}
        \item 速度控制:直接输出关节速度命令
        \item 力矩控制:输出关节力矩命令
    \end{itemize}
\end{enumerate}

\section{MuJoCo XML配置}

\subsection{panda.xml中的执行器配置}

速度控制版本将所有7个关节的执行器从\texttt{<motor>}改为\texttt{<velocity>}:

\begin{lstlisting}[language=XML, caption=速度执行器配置]
<actuator>
  <velocity ctrllimited="true" ctrlrange="-2.0 2.0" 
            joint="panda_joint1" kv="10" 
            name="vel_panda_joint1" />
  <velocity ctrllimited="true" ctrlrange="-2.0 2.0" 
            joint="panda_joint2" kv="10" 
            name="vel_panda_joint2" />
  <!-- ... 其他关节 ... -->
  <velocity ctrllimited="true" ctrlrange="-2.5 2.5" 
            joint="panda_joint5" kv="10" 
            name="vel_panda_joint5" />
  <!-- ... -->
</actuator>
\end{lstlisting}

\textbf{参数说明:}
\begin{itemize}
    \item \texttt{ctrlrange}:速度限制(rad/s),前4个关节为±2.0,后3个关节为±2.5
    \item \texttt{kv}:速度阻尼系数,提供速度反馈阻尼
    \item \texttt{name}:执行器名称,必须与代码中的\texttt{velocity\_actuator\_names}匹配
\end{itemize}

\section{使用说明}

\subsection{运行仿真}

\begin{lstlisting}[caption=运行速度控制仿真]
if __name__ == "__main__":
    sim = TaskSpaceVelocitySimulation(
        model_path="surface_force_control.xml"
    )
    sim.debug = False  # 设置为True启用调试输出
    sim.run(duration=30.0)
\end{lstlisting}

\subsection{调整控制参数}

可以通过修改\texttt{setup\_control\_parameters()}方法中的$K_p$矩阵来调整控制性能:

\begin{lstlisting}[caption=调整控制增益]
def setup_control_parameters(self):
    # 增大增益可以提高响应速度,但可能引起振荡
    # 减小增益可以提高稳定性,但响应较慢
    self.K_p = np.diag([5.0, 5.0, 5.0, 3.0, 3.0, 3.0])  # [旋转, 位置]
\end{lstlisting}

\section{数学公式总结}

\subsection{控制律}

\begin{equation}
\dot{\theta} = J^+(\theta)\left[\dot{X}_d + K_p X_e\right]
\end{equation}

\subsection{配置误差}

\begin{equation}
X_e = \log(X^{-1}X_d) = \log\left(\begin{bmatrix} R_curr^T R_ref & R_curr^T(p_ref - p_curr) \\ 0 & 1 \end{bmatrix}\right)
\end{equation}

\subsection{雅可比伪逆}

对于冗余机器人(7自由度,6维任务空间),使用伪逆:

\begin{equation}
J^+ = J^T(JJ^T)^{-1}
\end{equation}

或使用SVD分解的数值稳定方法(代码中使用\texttt{np.linalg.pinv})。

\section{总结}

本代码实现了基于公式236的任务空间速度控制,主要特点:

\begin{itemize}
    \item \textbf{简单性}:不需要动力学模型,仅需雅可比矩阵
    \item \textbf{效率}:计算量小,适合实时控制
    \item \textbf{6DOF扩展}:支持位置和姿态的联合控制
    \item \textbf{冗余处理}:使用伪逆处理7自由度冗余机器人
\end{itemize}

速度控制适用于具有速度控制执行器的机器人系统,相比力矩控制实现更简单,但控制精度可能略低,因为不利用动力学模型进行补偿。

\end{document}

