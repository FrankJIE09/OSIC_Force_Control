\documentclass[UTF8,12pt]{ctexart}
\usepackage{amsmath}
\usepackage{amssymb}
\usepackage{geometry}
\geometry{a4paper, margin=2.5cm}

\title{OSIC 表面力控仿真公式总结}
\author{基于 osic\_viewer.py}
\date{}

\begin{document}

\maketitle

\section{基础定义}

\subsection{系统参数}
\begin{itemize}
    \item $n = 7$:关节数量(Franka Panda 7-DOF)
    \item $\mathbf{q} \in \mathbb{R}^7$:关节角度向量
    \item $\dot{\mathbf{q}} \in \mathbb{R}^7$:关节角速度向量
    \item $\boldsymbol{\tau} \in \mathbb{R}^7$:关节力矩向量
    \item $\mathbf{p} \in \mathbb{R}^3$:末端执行器位置(笛卡尔空间)
    \item $\mathbf{v} \in \mathbb{R}^3$:末端执行器速度(笛卡尔空间)
    \item $\mathbf{F} \in \mathbb{R}^3$:末端执行器力(笛卡尔空间)
\end{itemize}

\subsection{雅可比矩阵}
末端执行器位置雅可比矩阵 $\mathbf{J} \in \mathbb{R}^{3 \times 7}$ 将关节速度映射到末端执行器速度:

\begin{equation}
\mathbf{v} = \mathbf{J} \dot{\mathbf{q}}
\end{equation}

其中 $\mathbf{J}$ 通过 MuJoCo 的 \texttt{mj\_jacBody} 函数计算得到。

\section{接触力检测}

接触力 $F_z$ 通过累加所有与表面接触点的法向力分量得到:

\begin{equation}
F_z = \sum_{i=1}^{n_{\text{con}}} f_{z,i}
\end{equation}

其中 $n_{\text{con}}$ 是当前接触点数量,$f_{z,i}$ 是第 $i$ 个接触点的 Z 方向力分量。

\subsection{接触状态滞后机制}
接触状态 $s_{\text{contact}}$ 采用滞后机制以避免抖动:

\begin{equation}
s_{\text{contact}} = \begin{cases}
\text{True} & \text{if } |F_z| > 1.0 \text{ N} \\
\text{False} & \text{if } |F_z| < 0.05 \text{ N} \\
s_{\text{contact}}^{\text{prev}} & \text{otherwise}
\end{cases}
\end{equation}

\section{阶段识别}

控制算法分为 4 个阶段,基于接触力和时间:

\begin{equation}
\text{phase} = \begin{cases}
0 & \text{if } |F_{\text{curr}}| < 1.0 \text{ N} \\
1 & \text{if } |F_{\text{curr}}| \geq 1.0 \text{ and } t_{\text{contact}} < 1.0 \text{ s} \\
2 & \text{if } 1.0 \text{ s} \leq t_{\text{contact}} < 3.0 \text{ s} \\
3 & \text{if } t_{\text{contact}} \geq 3.0 \text{ s}
\end{cases}
\end{equation}

其中 $t_{\text{contact}} = t - t_{\text{contact\_start}}$ 是自首次接触以来的时间。

\section{Z 轴控制力}

\subsection{Phase 0:快速下降}
目标位置随时间线性下降:

\begin{equation}
z_{\text{des}} = 0.4 - \min\left(\frac{t}{3.0}, 1.0\right) \times 0.25
\end{equation}

控制力采用 PD 控制:

\begin{equation}
F_{\text{cmd},z} = 300.0 \times (z_{\text{des}} - p_{z,\text{curr}}) - 20.0 \times v_{z,\text{curr}}
\end{equation}

\subsection{Phase 1:缓慢下降并产生压力}
目标位置缓慢下降:

\begin{equation}
z_{\text{des}} = \max\left(0.15 - 0.01 \times (t - t_{\text{contact}}), 0.1\right)
\end{equation}

控制力包含位置误差、速度反馈和常力项:

\begin{equation}
F_{\text{cmd},z} = 250.0 \times (z_{\text{des}} - p_{z,\text{curr}}) - 15.0 \times v_{z,\text{curr}} + 30.0
\end{equation}

\subsection{Phase 2:保持位置并产生稳定压力}
目标位置固定:

\begin{equation}
z_{\text{des}} = 0.12
\end{equation}

控制力:

\begin{equation}
F_{\text{cmd},z} = 200.0 \times (z_{\text{des}} - p_{z,\text{curr}}) - 10.0 \times v_{z,\text{curr}} + 40.0
\end{equation}

\subsection{Phase 3:维持接触}
目标位置锁定:

\begin{equation}
z_{\text{des}} = 0.24
\end{equation}

控制力(仅位置和速度反馈):

\begin{equation}
F_{\text{cmd},z} = 150.0 \times (z_{\text{des}} - p_{z,\text{curr}}) - 5.0 \times v_{z,\text{curr}}
\end{equation}

\section{XY 轴控制与切向运动}

\subsection{切向运动轨迹生成}

\subsubsection{X 轴前后擦拭($0 \leq t_{\text{wipe}} < 15.0$ s)}
擦拭时间:$t_{\text{wipe}} = t_{\text{contact}} - 8.0$(接触后 8 秒开始)

周期时间:$\tau = t_{\text{wipe}} \bmod 1.0$(1 秒一个周期)

当 $\tau < 0.5$ 时:
\begin{align}
\text{progress} &= \frac{\tau}{0.5} \\
p_{x,\text{ref}} &= 0.5 + 0.15 \times \sin(\pi \times \text{progress}) \\
v_{x,\text{ref}} &= 0.15 \times \frac{\pi}{0.5} \times \cos(\pi \times \text{progress})
\end{align}

当 $\tau \geq 0.5$ 时:
\begin{align}
\text{progress} &= \frac{\tau - 0.5}{0.5} \\
p_{x,\text{ref}} &= 0.5 + 0.15 \times \sin(\pi \times (1 - \text{progress})) \\
v_{x,\text{ref}} &= -0.15 \times \frac{\pi}{0.5} \times \cos(\pi \times (1 - \text{progress}))
\end{align}

\subsubsection{Y 轴左右擦拭($t_{\text{wipe}} \geq 15.0$ s)}
周期时间:$\tau = (t_{\text{wipe}} - 15.0) \bmod 1.0$

当 $\tau < 0.5$ 时:
\begin{align}
\text{progress} &= \frac{\tau}{0.5} \\
p_{y,\text{ref}} &= -0.1 \times \sin(\pi \times \text{progress}) \\
v_{y,\text{ref}} &= -0.1 \times \frac{\pi}{0.5} \times \cos(\pi \times \text{progress})
\end{align}

当 $\tau \geq 0.5$ 时:
\begin{align}
\text{progress} &= \frac{\tau - 0.5}{0.5} \\
p_{y,\text{ref}} &= -0.1 \times \sin(\pi \times (1 - \text{progress})) \\
v_{y,\text{ref}} &= 0.1 \times \frac{\pi}{0.5} \times \cos(\pi \times (1 - \text{progress}))
\end{align}

\subsection{XY 轴控制力}

位置误差:
\begin{equation}
\mathbf{e}_{xy} = \mathbf{p}_{xy,\text{ref}} - \mathbf{p}_{xy,\text{curr}}
\end{equation}

当接触且 $|F_{\text{curr}}| > 2.0$ N 时(使用速度前馈):
\begin{equation}
\mathbf{F}_{\text{cmd},xy} = 800.0 \times \mathbf{e}_{xy} + 2.0 \times \mathbf{v}_{xy,\text{ref}} - 3.0 \times \mathbf{v}_{xy,\text{curr}}
\end{equation}

否则(仅位置和速度反馈):
\begin{equation}
\mathbf{F}_{\text{cmd},xy} = 200.0 \times \mathbf{e}_{xy} - 15.0 \times \mathbf{v}_{xy,\text{curr}}
\end{equation}

\section{关节力矩计算}

\subsection{操作空间到关节空间映射}
使用雅可比转置将笛卡尔空间力映射到关节力矩:

\begin{equation}
\boldsymbol{\tau}_{\text{op}} = \mathbf{J}^T \mathbf{F}_{\text{cmd}}
\end{equation}

\subsection{重力补偿}
重力补偿力矩从 MuJoCo 的 \texttt{qfrc\_bias} 获取:

\begin{equation}
\boldsymbol{\tau}_{\text{bias}} = \mathbf{qfrc\_bias}
\end{equation}

总关节力矩:

\begin{equation}
\boldsymbol{\tau} = \boldsymbol{\tau}_{\text{bias}} + \boldsymbol{\tau}_{\text{op}}
\end{equation}

\subsection{力矩限制}
关节力矩被限制在安全范围内:

\begin{equation}
\boldsymbol{\tau}_{\max} = [87, 87, 87, 87, 12, 12, 12]^T \text{ N·m}
\end{equation}

\begin{equation}
\boldsymbol{\tau} = \text{clip}(\boldsymbol{\tau}, -\boldsymbol{\tau}_{\max}, \boldsymbol{\tau}_{\max})
\end{equation}

即:
\begin{equation}
\tau_i = \begin{cases}
-\tau_{\max,i} & \text{if } \tau_i < -\tau_{\max,i} \\
\tau_{\max,i} & \text{if } \tau_i > \tau_{\max,i} \\
\tau_i & \text{otherwise}
\end{cases}
\end{equation}

\section{控制参数总结}

\subsection{Z 轴控制参数}
\begin{table}[h]
\centering
\begin{tabular}{|c|c|c|c|c|}
\hline
阶段 & $K_p$ & $K_d$ & $F_{\text{offset}}$ & $z_{\text{des}}$ 公式 \\
\hline
Phase 0 & 300.0 & 20.0 & 0 & $0.4 - \min(t/3, 1) \times 0.25$ \\
Phase 1 & 250.0 & 15.0 & 30.0 & $\max(0.15 - 0.01(t-t_c), 0.1)$ \\
Phase 2 & 200.0 & 10.0 & 40.0 & $0.12$ \\
Phase 3 & 150.0 & 5.0 & 0 & $0.24$ \\
\hline
\end{tabular}
\caption{Z 轴控制参数}
\end{table}

\subsection{XY 轴控制参数}
\begin{table}[h]
\centering
\begin{tabular}{|c|c|c|c|}
\hline
状态 & $K_p$ & $K_d$ & $K_v$ (速度前馈) \\
\hline
接触 ($|F| > 2$ N) & 800.0 & 3.0 & 2.0 \\
非接触 & 200.0 & 15.0 & 0 \\
\hline
\end{tabular}
\caption{XY 轴控制参数}
\end{table}

\subsection{切向运动参数}
\begin{table}[h]
\centering
\begin{tabular}{|c|c|c|c|}
\hline
方向 & 幅度 (m) & 周期 (s) & 开始时间 \\
\hline
X 轴(前后) & 0.15 & 1.0 & $t_{\text{contact}} = 8.0$ s \\
Y 轴(左右) & 0.10 & 1.0 & $t_{\text{contact}} = 23.0$ s \\
\hline
\end{tabular}
\caption{切向运动参数}
\end{table}

\end{document}

