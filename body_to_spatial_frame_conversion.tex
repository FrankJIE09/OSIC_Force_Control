\documentclass[12pt,a4paper]{article}
\usepackage[UTF8]{ctex}
\usepackage{amsmath,amssymb,amsthm}
\usepackage{graphicx}
\usepackage{geometry}
\usepackage{hyperref}
\usepackage{amsthm}
\usepackage{bm}

\geometry{margin=2.5cm}

\title{从物体坐标系到空间坐标系的任务空间控制律转换}
\subtitle{配置误差$X_e$和速度误差$V_e$的详细推导}
\author{机器人控制理论}
\date{\today}

\newtheorem{definition}{定义}
\newtheorem{theorem}{定理}
\newtheorem{lemma}{引理}

\begin{document}

\maketitle

\section{引言}

在机器人任务空间控制中,我们可以在两种不同的坐标系中表示控制律:
\begin{itemize}
    \item \textbf{物体坐标系(Body Frame)}:固定在末端执行器上的坐标系$\{b\}$
    \item \textbf{空间坐标系(Spatial Frame)}:固定在世界坐标系中的坐标系$\{s\}$
\end{itemize}

本文档详细推导了从物体坐标系到空间坐标系的完整转换过程,特别关注配置误差$X_e$和速度误差$V_e$的计算方法。

\section{坐标系定义和基本关系}

\subsection{位姿表示}

设$X \in SE(3)$表示末端执行器相对于空间坐标系的位姿:
\begin{equation}
X = \begin{bmatrix} R & p \\ 0 & 1 \end{bmatrix}
\end{equation}
其中$R \in SO(3)$是旋转矩阵,$p \in \mathbb{R}^3$是位置向量。

\subsection{Twist的表示}

在物体坐标系$\{b\}$中,twist $V_b$定义为:
\begin{equation}
[V_b] = X^{-1}\dot{X} = \begin{bmatrix} [\omega_b]_\times & v_b \\ 0 & 0 \end{bmatrix}
\end{equation}
其中$[\omega_b]_\times$是角速度$\omega_b$的反对称矩阵,$v_b$是线速度。

在空间坐标系$\{s\}$中,twist $V_s$定义为:
\begin{equation}
[V_s] = \dot{X}X^{-1} = \begin{bmatrix} [\omega_s]_\times & v_s \\ 0 & 0 \end{bmatrix}
\end{equation}

\subsection{Twist的坐标变换}

物体坐标系和空间坐标系中的twist通过伴随变换关联:
\begin{equation}
V_s = [Ad_X] V_b
\end{equation}

\textbf{重要说明}:$[Ad_X]$中的$X$是$T_{s \to b}$(从空间坐标系到物体坐标系的变换),即:
\begin{equation}
X = T_{s \to b} = \begin{bmatrix} R & p \\ 0 & 1 \end{bmatrix}
\end{equation}

伴随变换矩阵为:
\begin{equation}
[Ad_X] = \begin{bmatrix} R & 0 \\ [p]_\times R & R \end{bmatrix}
\end{equation}
其中$[p]_\times$是位置向量$p$的反对称矩阵:
\begin{equation}
[p]_\times = \begin{bmatrix}
0 & -p_z & p_y \\
p_z & 0 & -p_x \\
-p_y & p_x & 0
\end{bmatrix}
\end{equation}

\subsection{Wrench的坐标变换}

类似地,wrench在两种坐标系中的关系为:
\begin{equation}
F_s = [Ad_X] F_b
\end{equation}
其中$F_b$是在物体坐标系中表示的wrench,$F_s$是在空间坐标系中表示的wrench。

\section{雅可比矩阵的转换}

\subsection{物体雅可比矩阵}

物体雅可比矩阵$J_b(\theta)$将关节速度映射到物体坐标系中的twist:
\begin{equation}
V_b = J_b(\theta) \dot{\theta}
\end{equation}

\subsection{空间雅可比矩阵}

空间雅可比矩阵$J_s(\theta)$将关节速度映射到空间坐标系中的twist:
\begin{equation}
V_s = J_s(\theta) \dot{\theta}
\end{equation}

\subsection{雅可比矩阵的转换关系}

由于$V_s = [Ad_X] V_b$,我们有:
\begin{align}
V_s &= [Ad_X] V_b \\
J_s(\theta) \dot{\theta} &= [Ad_X] J_b(\theta) \dot{\theta}
\end{align}
因此:
\begin{equation}
J_s(\theta) = [Ad_X] J_b(\theta)
\end{equation}
或者等价地:
\begin{equation}
J_b(\theta) = [Ad_{X^{-1}}] J_s(\theta)
\end{equation}

\section{任务空间动力学}

\subsection{物体坐标系中的动力学}

在物体坐标系中,任务空间动力学为:
\begin{equation}
F_b = \Lambda_b(\theta) \dot{V}_b + \eta_b(\theta, V_b)
\end{equation}
其中:
\begin{itemize}
    \item $\Lambda_b(\theta) = (J_b(\theta) M^{-1}(\theta) J_b^T(\theta))^{-1}$是任务空间质量矩阵
    \item $\eta_b(\theta, V_b)$是任务空间科里奥利项
\end{itemize}

关节力矩与wrench的关系为:
\begin{equation}
\tau = J_b^T(\theta) F_b
\end{equation}

\subsection{空间坐标系中的动力学}

在空间坐标系中,任务空间动力学为:
\begin{equation}
F_s = \Lambda_s(\theta) \dot{V}_s + \eta_s(\theta, V_s)
\end{equation}
其中:
\begin{itemize}
    \item $\Lambda_s(\theta) = (J_s(\theta) M^{-1}(\theta) J_s^T(\theta))^{-1}$是任务空间质量矩阵
    \item $\eta_s(\theta, V_s)$是任务空间科里奥利项
\end{itemize}

关节力矩与wrench的关系为:
\begin{equation}
\tau = J_s^T(\theta) F_s
\end{equation}

\subsection{动力学量的转换关系}

通过伴随变换,我们可以证明:
\begin{align}
\Lambda_s(\theta) &= [Ad_X] \Lambda_b(\theta) [Ad_X]^T \\
\eta_s(\theta, V_s) &= [Ad_X] \eta_b(\theta, V_b) + \Lambda_s(\theta) [Ad_X] \dot{V}_b - \Lambda_s(\theta) \dot{V}_s
\end{align}

\section{控制律的转换}

\subsection{物体坐标系中的控制律}

在物体坐标系中,任务空间控制律为:
\begin{equation}
\tau = J_b^T(\theta)\left(\tilde{\Lambda}_b(\theta)\frac{d}{dt}([Ad_{X^{-1}X_d}]V_d) + K_p X_e^b + K_i\int_0^t X_e^b(t)dt + K_d V_e^b\right) + \tilde{\eta}_b(\theta, V_b)
\end{equation}
其中上标$b$表示在物体坐标系中表示的量。

\subsection{空间坐标系中的控制律}

在空间坐标系中,对应的控制律为:
\begin{equation}
\tau = J_s^T(\theta)\left(\tilde{\Lambda}_s(\theta)\frac{d}{dt}([Ad_{X_d X^{-1}}]V_d) + K_p X_e^s + K_i\int_0^t X_e^s(t)dt + K_d V_e^s\right) + \tilde{\eta}_s(\theta, V_s)
\end{equation}
其中上标$s$表示在空间坐标系中表示的量。

\section{配置误差$X_e$的详细推导}

\subsection{坐标系变换的直观理解}

在推导配置误差之前,我们先理解坐标系变换的表示方法:

\textbf{变换矩阵的表示约定}:
\begin{itemize}
    \item $T_{A \to B}$:从坐标系$\{A\}$到坐标系$\{B\}$的变换矩阵
    \item 如果点$p$在坐标系$\{A\}$中的坐标为$p_A$,在坐标系$\{B\}$中的坐标为$p_B$,则:
    \begin{equation}
    p_B = T_{A \to B} \cdot p_A
    \end{equation}
\end{itemize}

\textbf{在我们的问题中}:
\begin{itemize}
    \item $\{s\}$:空间坐标系(固定在世界中)
    \item $\{b\}$:物体坐标系(固定在末端执行器上,当前配置)
    \item $\{d\}$:期望配置坐标系(期望的末端执行器配置)
\end{itemize}

\textbf{变换矩阵的定义}:
\begin{align}
X &= T_{s \to b} = \begin{bmatrix} R & p \\ 0 & 1 \end{bmatrix} \quad \text{(从空间到物体)} \\
X^{-1} &= T_{b \to s} = \begin{bmatrix} R^T & -R^T p \\ 0 & 1 \end{bmatrix} \quad \text{(从物体到空间)} \\
X_d &= T_{s \to d} = \begin{bmatrix} R_d & p_d \\ 0 & 1 \end{bmatrix} \quad \text{(从空间到期望配置)}
\end{align}

\textbf{关键理解}:
\begin{itemize}
    \item $X^{-1} X_d = T_{b \to s} \cdot T_{s \to d} = T_{b \to d}$:从物体坐标系到期望配置坐标系的变换(在物体坐标系中表示)
    \item $X_d X^{-1} = T_{s \to d} \cdot T_{b \to s} = T_{b \to d}$:从物体坐标系到期望配置坐标系的变换(在空间坐标系中表示)
\end{itemize}

注意:虽然两者都表示$T_{b \to d}$,但\textbf{表示的坐标系不同}!

\subsection{物体坐标系中的配置误差}

在物体坐标系中,配置误差定义为:
\begin{equation}
[X_e^b] = \log(X^{-1} X_d)
\end{equation}

\textbf{物理意义}:$X_e^b$是在末端执行器坐标系$\{b\}$中表示的twist,如果跟随单位时间,会将当前配置$X$移动到期望配置$X_d$。

\textbf{计算步骤}:

设$X = (R, p)$是当前位姿,$X_d = (R_d, p_d)$是期望位姿。

\begin{enumerate}
    \item 计算相对变换:
    \begin{align}
    X_{\text{err}} &= X^{-1} X_d \\
    &= \begin{bmatrix} R^T & -R^T p \\ 0 & 1 \end{bmatrix} \begin{bmatrix} R_d & p_d \\ 0 & 1 \end{bmatrix} \\
    &= \begin{bmatrix} R^T R_d & R^T(p_d - p) \\ 0 & 1 \end{bmatrix}
    \end{align}
    
    \item 提取旋转和平移部分:
    \begin{align}
    R_{\text{err}} &= R^T R_d \\
    p_{\text{err}} &= R^T(p_d - p)
    \end{align}
    
    \item 计算旋转部分(旋转向量):
    \begin{equation}
    \omega = \log(R_{\text{err}})
    \end{equation}
    其中$\log: SO(3) \to \mathfrak{so}(3)$是旋转群的log映射。
    
    如果$\theta = \|\omega\|$是旋转角度,$\hat{\omega} = \omega / \theta$是旋转轴,则:
    \begin{equation}
    \theta = \arccos\left(\frac{\text{tr}(R_{\text{err}}) - 1}{2}\right)
    \end{equation}
    当$\theta$很小时,可以使用近似:
    \begin{equation}
    \omega \approx \frac{1}{2}\begin{bmatrix} R_{32} - R_{23} \\ R_{13} - R_{31} \\ R_{21} - R_{12} \end{bmatrix}
    \end{equation}
    
    \item 计算平移部分:
    \begin{equation}
    v = V^{-1}(\omega) \cdot p_{\text{err}}
    \end{equation}
    其中$V^{-1}(\omega)$是$3 \times 3$矩阵:
    \begin{equation}
    V^{-1}(\omega) = I - \frac{1-\cos\theta}{\theta^2}[\hat{\omega}]_\times + \frac{\theta - \sin\theta}{\theta^3}[\hat{\omega}]_\times^2
    \end{equation}
    当$\theta$很小时,可以使用近似:
    \begin{equation}
    V^{-1}(\omega) \approx I - \frac{1}{2}[\hat{\omega}]_\times
    \end{equation}
    
    \item 组合为6维twist:
    \begin{equation}
    X_e^b = \begin{bmatrix} v \\ \omega \end{bmatrix}
    \end{equation}
\end{enumerate}

\subsection{空间坐标系中的配置误差}

在空间坐标系中,配置误差定义为:
\begin{equation}
[X_e^s] = \log(X_d X^{-1})
\end{equation}

\textbf{变换矩阵的物理意义}:

首先,我们需要理解各个变换矩阵的物理意义:
\begin{itemize}
    \item $X = T_{s \to b}$:从空间坐标系$\{s\}$到物体坐标系$\{b\}$的变换
    \item $X^{-1} = T_{b \to s}$:从物体坐标系$\{b\}$到空间坐标系$\{s\}$的变换($X$的逆)
    \item $X_d = T_{s \to d}$:从空间坐标系$\{s\}$到期望配置坐标系$\{d\}$的变换
\end{itemize}

\textbf{矩阵乘法的物理意义}:

$X_d X^{-1}$的乘法顺序表示复合变换:
\begin{equation}
X_d X^{-1} = T_{s \to d} \cdot T_{b \to s} = T_{b \to d}
\end{equation}

\textbf{解释}:
\begin{enumerate}
    \item 首先应用$X^{-1} = T_{b \to s}$:将点从物体坐标系$\{b\}$变换到空间坐标系$\{s\}$
    \item 然后应用$X_d = T_{s \to d}$:将点从空间坐标系$\{s\}$变换到期望配置坐标系$\{d\}$
    \item 结果$T_{b \to d}$:从物体坐标系$\{b\}$到期望配置坐标系$\{d\}$的变换
\end{enumerate}

\textbf{具体计算}:

设$X = (R, p)$表示从空间坐标系到物体坐标系的变换,$X_d = (R_d, p_d)$表示从空间坐标系到期望配置的变换。

则:
\begin{align}
X^{-1} &= \begin{bmatrix} R^T & -R^T p \\ 0 & 1 \end{bmatrix} = T_{b \to s} \\
X_d &= \begin{bmatrix} R_d & p_d \\ 0 & 1 \end{bmatrix} = T_{s \to d}
\end{align}

矩阵乘法:
\begin{align}
X_d X^{-1} &= \begin{bmatrix} R_d & p_d \\ 0 & 1 \end{bmatrix} \begin{bmatrix} R^T & -R^T p \\ 0 & 1 \end{bmatrix} \\
&= \begin{bmatrix} R_d R^T & R_d(-R^T p) + p_d \\ 0 & 1 \end{bmatrix} \\
&= \begin{bmatrix} R_d R^T & p_d - R_d R^T p \\ 0 & 1 \end{bmatrix}
\end{align}

这个结果$R_d R^T$和$p_d - R_d R^T p$表示从物体坐标系$\{b\}$到期望配置坐标系$\{d\}$的相对变换,但在\textbf{空间坐标系中表示}。

\textbf{具体例子说明}:

假设:
\begin{align}
X &= \begin{bmatrix} R & p \\ 0 & 1 \end{bmatrix} = \begin{bmatrix} I & \begin{bmatrix} 0 \\ 0 \\ 1 \end{bmatrix} \\ 0 & 1 \end{bmatrix} \quad \text{(物体在空间坐标系中z=1的位置)} \\
X_d &= \begin{bmatrix} R_d & p_d \\ 0 & 1 \end{bmatrix} = \begin{bmatrix} I & \begin{bmatrix} 1 \\ 0 \\ 1 \end{bmatrix} \\ 0 & 1 \end{bmatrix} \quad \text{(期望位置在空间坐标系中x=1, z=1)}
\end{align}

则:
\begin{align}
X^{-1} &= \begin{bmatrix} I & \begin{bmatrix} 0 \\ 0 \\ -1 \end{bmatrix} \\ 0 & 1 \end{bmatrix} \\
X_d X^{-1} &= \begin{bmatrix} I & \begin{bmatrix} 1 \\ 0 \\ 1 \end{bmatrix} \\ 0 & 1 \end{bmatrix} \begin{bmatrix} I & \begin{bmatrix} 0 \\ 0 \\ -1 \end{bmatrix} \\ 0 & 1 \end{bmatrix} \\
&= \begin{bmatrix} I & \begin{bmatrix} 1 \\ 0 \\ 0 \end{bmatrix} \\ 0 & 1 \end{bmatrix}
\end{align}

结果表示:从物体坐标系到期望配置坐标系的相对位置是$\begin{bmatrix} 1 \\ 0 \\ 0 \end{bmatrix}$(在空间坐标系中表示),即物体需要沿空间坐标系的x轴移动1个单位。

\textbf{物理意义}:$X_e^s$是在空间坐标系$\{s\}$中表示的twist,如果跟随单位时间,会将当前配置$X$移动到期望配置$X_d$。

\textbf{计算步骤}:

\begin{enumerate}
    \item 计算相对变换:
    \begin{align}
    X_{\text{err}} &= X_d X^{-1} \\
    &= \begin{bmatrix} R_d & p_d \\ 0 & 1 \end{bmatrix} \begin{bmatrix} R^T & -R^T p \\ 0 & 1 \end{bmatrix} \\
    &= \begin{bmatrix} R_d R^T & p_d - R_d R^T p \\ 0 & 1 \end{bmatrix}
    \end{align}
    
    \item 提取旋转和平移部分:
    \begin{align}
    R_{\text{err}} &= R_d R^T \\
    p_{\text{err}} &= p_d - R_d R^T p
    \end{align}
    
    \item 计算旋转部分(旋转向量):
    \begin{equation}
    \omega = \log(R_{\text{err}})
    \end{equation}
    计算方法与物体坐标系中相同。
    
    \item 计算平移部分:
    \begin{equation}
    v = V^{-1}(\omega) \cdot p_{\text{err}}
    \end{equation}
    其中$V^{-1}(\omega)$的定义与物体坐标系中相同。
    
    \item 组合为6维twist:
    \begin{equation}
    X_e^s = \begin{bmatrix} v \\ \omega \end{bmatrix}
    \end{equation}
\end{enumerate}

\subsection{两种表示的关系}

物体坐标系和空间坐标系中的配置误差通过伴随变换关联:
\begin{equation}
X_e^s = [Ad_X] X_e^b
\end{equation}

\textbf{关键说明}:
\begin{itemize}
    \item $[Ad_X]$中的$X$是$T_{s \to b}$(从空间坐标系到物体坐标系的变换)
    \item $X = \begin{bmatrix} R & p \\ 0 & 1 \end{bmatrix}$,其中$R$和$p$表示物体坐标系相对于空间坐标系的旋转和位置
    \item $[Ad_X]$将物体坐标系$\{b\}$中的量转换到空间坐标系$\{s\}$中
    \item 因此:$X_e^s = [Ad_X] X_e^b$ 表示将物体坐标系中的配置误差转换到空间坐标系中
\end{itemize}

\textbf{记忆方法}:
\begin{enumerate}
    \item $X = T_{s \to b}$:从空间到物体(这是$X$的定义)
    \item $[Ad_X]$:使用$X$构造的伴随变换矩阵
    \item $[Ad_X]$的作用:将物体坐标系中的量转换到空间坐标系中
    \item 因此:$X_e^s = [Ad_X] X_e^b$ 表示"空间坐标系中的误差 = 伴随变换 × 物体坐标系中的误差"
\end{enumerate}

\textbf{注意}:虽然$X = T_{s \to b}$是从空间到物体的变换,但$[Ad_X]$的作用是将物体坐标系中的量转换到空间坐标系中。这是伴随变换的性质。

\textbf{对比总结}:

\begin{table}[h]
\centering
\caption{物体坐标系与空间坐标系中配置误差的对比}
\begin{tabular}{|l|l|l|}
\hline
\textbf{项目} & \textbf{物体坐标系} & \textbf{空间坐标系} \\
\hline
公式 & $X_e^b = \log(X^{-1} X_d)$ & $X_e^s = \log(X_d X^{-1})$ \\
\hline
变换链 & $T_{b \to s} \cdot T_{s \to d} = T_{b \to d}$ & $T_{s \to d} \cdot T_{b \to s} = T_{b \to d}$ \\
\hline
矩阵形式 & $\log\left(\begin{bmatrix} R^T R_d & R^T(p_d-p) \\ 0 & 1 \end{bmatrix}\right)$ & $\log\left(\begin{bmatrix} R_d R^T & p_d-R_d R^T p \\ 0 & 1 \end{bmatrix}\right)$ \\
\hline
旋转部分 & $R_{\text{err}} = R^T R_d$ & $R_{\text{err}} = R_d R^T$ \\
\hline
平移部分 & $p_{\text{err}} = R^T(p_d - p)$ & $p_{\text{err}} = p_d - R_d R^T p$ \\
\hline
表示坐标系 & 在物体坐标系$\{b\}$中表示 & 在空间坐标系$\{s\}$中表示 \\
\hline
关系 & $X_e^b$ & $X_e^s = [Ad_X] X_e^b$ \\
\hline
\end{tabular}
\end{table}

\textbf{关键理解}:

虽然$X^{-1} X_d$和$X_d X^{-1}$在数学上表示不同的矩阵,但它们都描述了\textbf{同一个相对变换}$T_{b \to d}$(从物体坐标系到期望配置坐标系)。区别在于:
\begin{itemize}
    \item $X^{-1} X_d$:先变换到空间坐标系,再变换到期望配置,结果在\textbf{物体坐标系}中表示
    \item $X_d X^{-1}$:先变换到空间坐标系,再变换到期望配置,结果在\textbf{空间坐标系}中表示
\end{itemize}

通过伴随变换$[Ad_X]$,可以将一种表示转换为另一种表示。

\textbf{证明}:

我们知道:
\begin{align}
X_e^b &= \log(X^{-1} X_d) \\
X_e^s &= \log(X_d X^{-1})
\end{align}

由于$\log(X_d X^{-1}) = [Ad_X] \log(X^{-1} X_d)$(这是SE(3)群的性质),我们有:
\begin{equation}
X_e^s = [Ad_X] X_e^b
\end{equation}

\section{速度误差$V_e$的详细推导}

\subsection{物体坐标系中的速度误差}

在物体坐标系中,速度误差定义为:
\begin{equation}
V_e^b = [Ad_{X^{-1}X_d}] V_d - V_b
\end{equation}

\textbf{物理意义}:
\begin{itemize}
    \item $V_d$是在期望配置$X_d$的坐标系中表示的参考速度
    \item $[Ad_{X^{-1}X_d}] V_d$将参考速度从坐标系$X_d$转换到当前坐标系$X$
    \item $V_b$是在当前坐标系$X$中表示的实际速度
    \item 两者相减得到速度误差
\end{itemize}

\textbf{计算步骤}:

设$X = (R, p)$是当前位姿,$X_d = (R_d, p_d)$是期望位姿。

\begin{enumerate}
    \item 计算相对变换:
    \begin{equation}
    X_{\text{rel}} = X^{-1} X_d = (R^T R_d, R^T(p_d - p))
    \end{equation}
    
    \item 计算伴随变换矩阵:
    \begin{equation}
    [Ad_{X^{-1}X_d}] = \begin{bmatrix}
    R^T R_d & 0 \\
    [R^T(p_d - p)]_\times R^T R_d & R^T R_d
    \end{bmatrix}
    \end{equation}
    
    \item 转换参考速度:
    \begin{equation}
    V_d^{\text{transformed}} = [Ad_{X^{-1}X_d}] V_d
    \end{equation}
    
    \item 计算速度误差:
    \begin{equation}
    V_e^b = V_d^{\text{transformed}} - V_b
    \end{equation}
\end{enumerate}

\subsection{空间坐标系中的速度误差}

在空间坐标系中,速度误差定义为:
\begin{equation}
V_e^s = V_d - [Ad_{X X_d^{-1}}] V_s
\end{equation}

或者,如果参考速度$V_d$也在空间坐标系中表示,则更简单的形式为:
\begin{equation}
V_e^s = V_d - V_s
\end{equation}

\textbf{物理意义}:
\begin{itemize}
    \item 如果$V_d$和$V_s$都在空间坐标系中表示,则直接相减即可
    \item 如果$V_d$在期望配置$X_d$的坐标系中表示,则需要先转换到空间坐标系
\end{itemize}

\textbf{计算步骤}:

\begin{enumerate}
    \item 如果$V_d$在空间坐标系中表示:
    \begin{equation}
    V_e^s = V_d - V_s
    \end{equation}
    
    \item 如果$V_d$在期望配置$X_d$的坐标系中表示:
    \begin{enumerate}
        \item 计算相对变换:
        \begin{equation}
        X_{\text{rel}} = X X_d^{-1} = (R R_d^T, p - R R_d^T p_d)
        \end{equation}
        
        \item 计算伴随变换矩阵:
        \begin{equation}
        [Ad_{X X_d^{-1}}] = \begin{bmatrix}
        R R_d^T & 0 \\
        [p - R R_d^T p_d]_\times R R_d^T & R R_d^T
        \end{bmatrix}
        \end{equation}
        
        \item 转换实际速度:
        \begin{equation}
        V_s^{\text{transformed}} = [Ad_{X X_d^{-1}}] V_s
        \end{equation}
        
        \item 计算速度误差:
        \begin{equation}
        V_e^s = V_d - V_s^{\text{transformed}}
        \end{equation}
    \end{enumerate}
\end{enumerate}

\subsection{两种表示的关系}

物体坐标系和空间坐标系中的速度误差通过伴随变换关联:
\begin{equation}
V_e^s = [Ad_X] V_e^b
\end{equation}

\textbf{证明}:

从物体坐标系中的定义:
\begin{align}
V_e^b &= [Ad_{X^{-1}X_d}] V_d - V_b
\end{align}

应用伴随变换:
\begin{align}
[Ad_X] V_e^b &= [Ad_X] ([Ad_{X^{-1}X_d}] V_d - V_b) \\
&= [Ad_X][Ad_{X^{-1}X_d}] V_d - [Ad_X] V_b \\
&= [Ad_{X \cdot X^{-1} \cdot X_d}] V_d - V_s \\
&= [Ad_{X_d}] V_d - V_s
\end{align}

如果$V_d$在空间坐标系中表示(即$V_d = [Ad_{X_d}] V_d^b$),则:
\begin{equation}
V_e^s = V_d - V_s = [Ad_X] V_e^b
\end{equation}

\section{前馈加速度项的转换}

\subsection{物体坐标系中的前馈项}

在物体坐标系中,前馈加速度项为:
\begin{equation}
\frac{d}{dt}([Ad_{X^{-1}X_d}] V_d)
\end{equation}

这个项表示在物体坐标系中表示的期望加速度。

\subsection{空间坐标系中的前馈项}

在空间坐标系中,对应的前馈加速度项为:
\begin{equation}
\frac{d}{dt}([Ad_{X_d X^{-1}}] V_d)
\end{equation}

或者,如果$V_d$在空间坐标系中表示,则更简单的形式为:
\begin{equation}
\dot{V}_d
\end{equation}

\section{完整公式对比}

\subsection{物体坐标系中的完整控制律}

\begin{equation}
\tau = J_b^T(\theta)\left(\tilde{\Lambda}_b(\theta)\frac{d}{dt}([Ad_{X^{-1}X_d}]V_d) + K_p X_e^b + K_i\int_0^t X_e^b(t)dt + K_d V_e^b\right) + \tilde{\eta}_b(\theta, V_b)
\end{equation}

其中:
\begin{align}
X_e^b &= \log(X^{-1} X_d) \\
V_e^b &= [Ad_{X^{-1}X_d}] V_d - V_b \\
V_b &= J_b(\theta) \dot{\theta}
\end{align}

\subsection{空间坐标系中的完整控制律}

\begin{equation}
\tau = J_s^T(\theta)\left(\tilde{\Lambda}_s(\theta)\frac{d}{dt}([Ad_{X_d X^{-1}}]V_d) + K_p X_e^s + K_i\int_0^t X_e^s(t)dt + K_d V_e^s\right) + \tilde{\eta}_s(\theta, V_s)
\end{equation}

其中:
\begin{align}
X_e^s &= \log(X_d X^{-1}) \\
V_e^s &= V_d - V_s \quad \text{(如果$V_d$在空间坐标系中)} \\
V_s &= J_s(\theta) \dot{\theta}
\end{align}

\section{转换关系总结}

\begin{table}[h]
\centering
\caption{物体坐标系与空间坐标系转换关系总结}
\begin{tabular}{|l|l|l|}
\hline
\textbf{量} & \textbf{物体坐标系} & \textbf{空间坐标系} \\
\hline
Twist & $V_b = J_b \dot{\theta}$ & $V_s = J_s \dot{\theta}$ \\
\hline
Wrench & $F_b$ & $F_s = [Ad_X] F_b$ \\
\hline
雅可比矩阵 & $J_b$ & $J_s = [Ad_X] J_b$ \\
\hline
质量矩阵 & $\Lambda_b$ & $\Lambda_s = [Ad_X] \Lambda_b [Ad_X]^T$ \\
\hline
配置误差 & $X_e^b = \log(X^{-1} X_d)$ & $X_e^s = \log(X_d X^{-1})$ \\
\hline
速度误差 & $V_e^b = [Ad_{X^{-1}X_d}] V_d - V_b$ & $V_e^s = V_d - V_s$ \\
\hline
前馈加速度 & $\frac{d}{dt}([Ad_{X^{-1}X_d}] V_d)$ & $\dot{V}_d$ \\
\hline
\end{tabular}
\end{table}

\section{结论}

本文档详细推导了从物体坐标系到空间坐标系的任务空间控制律转换,重点说明了:

\begin{enumerate}
    \item 配置误差$X_e$在两种坐标系中的计算方法:
    \begin{itemize}
        \item 物体坐标系:$X_e^b = \log(X^{-1} X_d)$
        \item 空间坐标系:$X_e^s = \log(X_d X^{-1})$
        \item 关系:$X_e^s = [Ad_X] X_e^b$
    \end{itemize}
    
    \item 速度误差$V_e$在两种坐标系中的计算方法:
    \begin{itemize}
        \item 物体坐标系:$V_e^b = [Ad_{X^{-1}X_d}] V_d - V_b$
        \item 空间坐标系:$V_e^s = V_d - V_s$(当$V_d$在空间坐标系中)
        \item 关系:$V_e^s = [Ad_X] V_e^b$
    \end{itemize}
    
    \item 所有相关量通过伴随变换$[Ad_X]$关联
    \item 两种表示在数学上等价,选择哪种取决于具体应用场景
\end{enumerate}

\end{document}

