\documentclass[12pt]{article}
\usepackage[utf8]{inputenc}
\usepackage[T1]{fontenc}
\usepackage{amsmath}
\usepackage{amssymb}
\usepackage{amsthm}
\usepackage{geometry}
\geometry{a4paper, margin=2.5cm}

\title{SE(3)的Log映射中平移部分$v$的计算公式推导}
\author{机器人控制理论}
\date{\today}

\begin{document}

\maketitle

\section{问题背景}

在机器人控制中,我们需要计算配置误差$X_e = \log(X^{-1} X_d)$,其中$X, X_d \in \text{SE}(3)$是刚体变换。对于SE(3)中的变换:
\begin{equation}
T = \begin{bmatrix} R & p \\ 0 & 1 \end{bmatrix} \in \text{SE}(3)
\end{equation}
其中$R \in \text{SO}(3)$是旋转矩阵,$p \in \mathbb{R}^3$是平移向量。

SE(3)的log映射定义为:
\begin{equation}
\log(T) = \begin{bmatrix} v \\ \omega \end{bmatrix} \in \mathbb{R}^6
\end{equation}
其中:
\begin{itemize}
    \item $\omega \in \mathbb{R}^3$是旋转部分(旋转向量),通过$\omega = \log(R)$计算
    \item $v \in \mathbb{R}^3$是平移部分,需要通过特定公式计算
\end{itemize}

\section{平移部分$v$的计算公式}

平移部分$v$的计算公式为:
\begin{equation}
v = V^{-1}(\omega) \cdot p
\end{equation}
其中$V^{-1}(\omega)$是一个$3 \times 3$矩阵,定义为:
\begin{equation}
V^{-1}(\omega) = I - \frac{1-\cos(\theta)}{\theta^2} [\hat{\omega}]_\times + \frac{\theta - \sin(\theta)}{\theta^3} [\hat{\omega}]_\times^2
\end{equation}
其中:
\begin{itemize}
    \item $\theta = \|\omega\|$是旋转角度
    \item $\hat{\omega} = \omega / \theta$是归一化的旋转轴(当$\theta \neq 0$时)
    \item $[\hat{\omega}]_\times$是$\hat{\omega}$的反对称矩阵:
    \begin{equation}
    [\hat{\omega}]_\times = \begin{bmatrix}
        0 & -\hat{\omega}_z & \hat{\omega}_y \\
        \hat{\omega}_z & 0 & -\hat{\omega}_x \\
        -\hat{\omega}_y & \hat{\omega}_x & 0
    \end{bmatrix}
    \end{equation}
    \item $[\hat{\omega}]_\times^2 = [\hat{\omega}]_\times \cdot [\hat{\omega}]_\times$是反对称矩阵的平方
\end{itemize}

\section{公式推导}

\subsection{SE(3)的李代数结构}

SE(3)的李代数$\mathfrak{se}(3)$由所有形如:
\begin{equation}
\xi = \begin{bmatrix} [\omega]_\times & v \\ 0 & 0 \end{bmatrix} \in \mathfrak{se}(3)
\end{equation}
的矩阵组成,其中$\omega, v \in \mathbb{R}^3$。

对于SE(3)中的变换$T = [R, p; 0, 1]$,其log映射$\log(T) = [v, \omega]^T$满足:
\begin{equation}
T = \exp(\xi) = \exp\left(\begin{bmatrix} [\omega]_\times & v \\ 0 & 0 \end{bmatrix}\right)
\end{equation}

\subsection{指数映射的级数展开}

SE(3)的指数映射可以展开为:
\begin{equation}
\exp(\xi) = I + \xi + \frac{\xi^2}{2!} + \frac{\xi^3}{3!} + \cdots
\end{equation}

对于$\xi = \begin{bmatrix} [\omega]_\times & v \\ 0 & 0 \end{bmatrix}$,通过级数展开和矩阵运算,可以得到:
\begin{equation}
\exp(\xi) = \begin{bmatrix} \exp([\omega]_\times) & V(\omega) v \\ 0 & 1 \end{bmatrix}
\end{equation}
其中:
\begin{itemize}
    \item $\exp([\omega]_\times) = R$是旋转矩阵(Rodrigues公式)
    \item $V(\omega)$是$3 \times 3$矩阵,定义为:
    \begin{equation}
    V(\omega) = I + \frac{1-\cos(\theta)}{\theta^2} [\hat{\omega}]_\times + \frac{\theta - \sin(\theta)}{\theta^3} [\hat{\omega}]_\times^2
    \end{equation}
\end{itemize}

\subsection{从指数映射到log映射}

给定$T = [R, p; 0, 1]$,我们需要找到$\xi$使得$\exp(\xi) = T$。

从指数映射的表达式:
\begin{equation}
T = \begin{bmatrix} R & V(\omega) v \\ 0 & 1 \end{bmatrix} = \begin{bmatrix} R & p \\ 0 & 1 \end{bmatrix}
\end{equation}
可以得到:
\begin{equation}
p = V(\omega) v
\end{equation}

因此,要计算$v$,我们需要求$V(\omega)$的逆:
\begin{equation}
v = V^{-1}(\omega) p
\end{equation}

\subsection{$V^{-1}(\omega)$的推导}

$V(\omega)$的表达式为:
\begin{equation}
V(\omega) = I + \frac{1-\cos(\theta)}{\theta^2} [\hat{\omega}]_\times + \frac{\theta - \sin(\theta)}{\theta^3} [\hat{\omega}]_\times^2
\end{equation}

利用反对称矩阵的性质:
\begin{itemize}
    \item $[\hat{\omega}]_\times^3 = -[\hat{\omega}]_\times$(对于单位向量$\hat{\omega}$)
    \item $[\hat{\omega}]_\times^4 = -[\hat{\omega}]_\times^2$
    \item $[\hat{\omega}]_\times^5 = [\hat{\omega}]_\times$
    \item 以此类推,形成周期为4的循环
\end{itemize}

通过矩阵求逆或级数展开,可以得到$V^{-1}(\omega)$的表达式:
\begin{equation}
V^{-1}(\omega) = I - \frac{1-\cos(\theta)}{\theta^2} [\hat{\omega}]_\times + \frac{\theta - \sin(\theta)}{\theta^3} [\hat{\omega}]_\times^2
\end{equation}

\subsection{小角度近似}

当$\theta \ll 1$(小角度)时,可以使用Taylor展开:
\begin{align}
\cos(\theta) &\approx 1 - \frac{\theta^2}{2} + O(\theta^4) \\
\sin(\theta) &\approx \theta - \frac{\theta^3}{6} + O(\theta^5)
\end{align}

代入$V^{-1}(\omega)$的公式:
\begin{align}
V^{-1}(\omega) &\approx I - \frac{\theta^2/2}{\theta^2} [\hat{\omega}]_\times + \frac{\theta^3/6}{\theta^3} [\hat{\omega}]_\times^2 \\
&= I - \frac{1}{2} [\hat{\omega}]_\times + \frac{1}{6} [\hat{\omega}]_\times^2
\end{align}

对于非常小的角度,可以进一步近似为:
\begin{equation}
V^{-1}(\omega) \approx I - \frac{1}{2} [\omega]_\times
\end{equation}
因为当$\theta \to 0$时,$[\hat{\omega}]_\times^2$项可以忽略。

\section{代码实现}

在代码中,我们使用以下实现:

\begin{verbatim}
theta = ||omega||
if theta < 1e-6:
    # 小角度近似
    V_inv = I - 0.5 * [omega]_×
else:
    # 精确计算
    omega_normalized = omega / theta
    V_inv = I - (1-cos(theta))/theta² * [omega_normalized]_× 
            + (theta-sin(theta))/theta³ * [omega_normalized]_ײ
v = V_inv @ p
\end{verbatim}

\section{几何意义}

$V^{-1}(\omega)$矩阵的作用是将平移向量$p$从SE(3)变换的"原始"坐标系转换到twist空间中的平移部分$v$。

几何上,$v$表示的是:如果沿着旋转轴$\omega$旋转角度$\theta$,同时沿着某个方向平移,那么平移向量$p$在twist空间中的表示就是$v$。

$V^{-1}(\omega)$矩阵考虑了旋转对平移的影响,因为SE(3)中的旋转和平移是耦合的(不是简单的直积)。

\section{总结}

SE(3)的log映射中平移部分$v$的计算公式:
\begin{equation}
v = V^{-1}(\omega) p
\end{equation}
其中:
\begin{equation}
V^{-1}(\omega) = I - \frac{1-\cos(\theta)}{\theta^2} [\hat{\omega}]_\times + \frac{\theta - \sin(\theta)}{\theta^3} [\hat{\omega}]_\times^2
\end{equation}

这个公式来自于SE(3)的李群-李代数对应关系,确保了从SE(3)到$\mathfrak{se}(3)$的映射是正确且唯一的。

\end{document}



