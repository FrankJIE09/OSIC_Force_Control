\documentclass[12pt,a4paper]{article}
\usepackage[UTF8]{ctex}
\usepackage{amsmath}
\usepackage{amssymb}
\usepackage{geometry}
\geometry{margin=2.5cm}

\title{擦桌子任务解答:混合力位控制公式推导}
\author{解答}
\date{}

\begin{document}

\maketitle

\section{约束矩阵 $A(\theta)$(公式11.57)}

对于表面接触任务,约束条件为:
\begin{itemize}
    \item 法向平移:$v_z = 0$
    \item 所有旋转:$\omega_x = \omega_y = \omega_z = 0$
\end{itemize}

约束矩阵 $A(\theta) \in \mathbb{R}^{4 \times 6}$ 定义为:
\begin{equation}
A(\theta) V = 0
\end{equation}

其中 $V = [v_x, v_y, v_z, \omega_x, \omega_y, \omega_z]^T$ 是末端执行器坐标系中的twist。

对于完全约束(4个约束),约束矩阵为:
\begin{equation}
A = \begin{bmatrix}
0 & 0 & 1 & 0 & 0 & 0 \\  % v_z = 0
0 & 0 & 0 & 1 & 0 & 0 \\  % ω_x = 0
0 & 0 & 0 & 0 & 1 & 0 \\  % ω_y = 0
0 & 0 & 0 & 0 & 0 & 1     % ω_z = 0
\end{bmatrix}
\end{equation}

即:
\begin{equation}
A = \begin{bmatrix}
\mathbf{0}_{1 \times 2} & 1 & \mathbf{0}_{1 \times 3} \\
\mathbf{0}_{3 \times 3} & \mathbf{I}_3
\end{bmatrix}
\end{equation}

\section{投影矩阵 $P(\theta)$(公式11.63)}

投影矩阵将任务空间wrench投影到运动子空间(切向运动方向):
\begin{equation}
P = I - A^T(A\Lambda^{-1}A^T)^{-1}A\Lambda^{-1}
\end{equation}

其中:
\begin{itemize}
    \item $I \in \mathbb{R}^{6 \times 6}$:单位矩阵
    \item $\Lambda(\theta) \in \mathbb{R}^{6 \times 6}$:任务空间质量矩阵
    \item $A \in \mathbb{R}^{4 \times 6}$:约束矩阵
\end{itemize}

投影矩阵的性质:
\begin{itemize}
    \item $P$ 的秩为 $n-k = 6-4 = 2$(运动子空间维度)
    \item $I-P$ 的秩为 $k = 4$(力子空间维度)
    \item $P$ 和 $I-P$ 正交:$P(I-P) = 0$
\end{itemize}

\section{配置误差 $X_e$(SE(3) log映射)}

配置误差定义为:
\begin{equation}
X_e = \log(X^{-1} X_d)
\end{equation}

其中:
\begin{itemize}
    \item $X = (R_{\text{curr}}, p_{\text{curr}})$:当前位姿
    \item $X_d = (R_{\text{ref}}, p_{\text{ref}})$:参考位姿
    \item $\log: SE(3) \to \mathfrak{se}(3)$:SE(3)群的log映射
\end{itemize}

计算步骤:
\begin{enumerate}
    \item 计算相对变换:$X_{\text{err}} = X^{-1} X_d = (R_{\text{curr}}^T R_{\text{ref}}, R_{\text{curr}}^T(p_{\text{ref}} - p_{\text{curr}}))$
    \item 提取旋转矩阵:$R_{\text{err}} = R_{\text{curr}}^T R_{\text{ref}}$
    \item 提取平移向量:$p_{\text{err}} = R_{\text{curr}}^T(p_{\text{ref}} - p_{\text{curr}})$
    \item 计算旋转部分:$\omega = \log(R_{\text{err}})$(旋转向量,轴角表示)
    \item 计算平移部分:$v = V^{-1}(\omega) p_{\text{err}}$,其中
    \begin{equation}
    V^{-1}(\omega) = I - \frac{1-\cos\theta}{\theta^2}[\omega]_\times + \frac{\theta-\sin\theta}{\theta^3}[\omega]_\times^2
    \end{equation}
    当 $\theta = \|\omega\|$ 很小时,使用近似:$V^{-1} \approx I - \frac{1}{2}[\omega]_\times$
    \item 组合为6维twist:$X_e = [v^T, \omega^T]^T$
\end{enumerate}

\section{速度误差 $V_e$}

速度误差定义为:
\begin{equation}
V_e = [Ad_{X^{-1}X_d}]V_d - V
\end{equation}

其中:
\begin{itemize}
    \item $V_d = [v_d^T, \omega_d^T]^T$:参考速度(在参考坐标系中表示)
    \item $V = [v^T, \omega^T]^T$:当前速度(在当前坐标系中表示)
    \item $[Ad_{X^{-1}X_d}]$:Adjoint变换,将$V_d$从参考坐标系转换到当前坐标系
\end{itemize}

Adjoint变换矩阵:
\begin{equation}
Ad_X = \begin{bmatrix}
R & 0 \\
[p]_\times R & R
\end{bmatrix}
\end{equation}

其中 $[p]_\times$ 是位置向量 $p$ 的反对称矩阵:
\begin{equation}
[p]_\times = \begin{bmatrix}
0 & -p_z & p_y \\
p_z & 0 & -p_x \\
-p_y & p_x & 0
\end{bmatrix}
\end{equation}

对于 $X_{\text{err}} = X^{-1} X_d = (R_{\text{err}}, p_{\text{err}})$:
\begin{equation}
[Ad_{X^{-1}X_d}] = \begin{bmatrix}
R_{\text{err}} & 0 \\
[p_{\text{err}}]_\times R_{\text{err}} & R_{\text{err}}
\end{bmatrix}
\end{equation}

\section{任务空间质量矩阵 $\tilde{\Lambda}(\theta)$}

任务空间质量矩阵定义为:
\begin{equation}
\tilde{\Lambda}(\theta) = (J_b(\theta) M^{-1}(\theta) J_b^T(\theta))^{-1}
\end{equation}

其中:
\begin{itemize}
    \item $M(\theta) \in \mathbb{R}^{7 \times 7}$:关节空间质量矩阵
    \item $J_b(\theta) \in \mathbb{R}^{6 \times 7}$:末端执行器坐标系中的雅可比矩阵
\end{itemize}

或者使用伪逆形式(当$J_b$不是方阵时):
\begin{equation}
\tilde{\Lambda}(\theta) = J_b^{-T}(\theta) M(\theta) J_b^{-1}(\theta)
\end{equation}

其中 $J_b^{-1} = (J_b^T J_b + \epsilon I)^{-1} J_b^T$ 是阻尼伪逆。

\section{任务空间Coriolis项 $\tilde{\eta}(\theta, V_b)$}

\subsection{代码中实际使用的公式(公式8.91)}

代码中实际使用的是公式8.91:
\begin{equation}
\tilde{\eta}(\theta, V_b) = J_b^{-T}(\theta) h(\theta, J_b^{-1}V_b) - \Lambda(\theta) \dot{J}_b(\theta) J_b^{-1}(\theta) V_b \tag{8.91}
\end{equation}

其中:
\begin{itemize}
    \item $h(\theta, \dot{\theta}) = C(\theta, \dot{\theta}) \dot{\theta} + g(\theta)$:关节空间的科里奥利、向心力和重力项
    \item $\Lambda(\theta) = J_b^{-T}(\theta) M(\theta) J_b^{-1}(\theta)$:任务空间质量矩阵
    \item $\dot{J}_b(\theta)$:雅可比矩阵的时间导数
    \item $V_b = J_b(\theta) \dot{\theta}$:末端执行器坐标系中的速度
    \item $\dot{\theta} = J_b^{-1}(\theta) V_b$:从任务空间速度反算的关节速度
\end{itemize}

\subsection{另一种等价形式}

另一种常见的等价形式为:
\begin{equation}
\tilde{\eta}(\theta, V_b) = J_b^{-T}(\theta) \left(C(\theta, \dot{\theta}) \dot{\theta} + g(\theta) - M(\theta) J_b^{-1}(\theta) \dot{J}_b(\theta) V_b\right)
\end{equation}

\subsection{两种形式的等价性证明}

将公式8.91展开:
\begin{align}
\tilde{\eta}(\theta, V_b) &= J_b^{-T}(\theta) h(\theta, J_b^{-1}V_b) - \Lambda(\theta) \dot{J}_b(\theta) J_b^{-1}(\theta) V_b \\
&= J_b^{-T}(\theta) \left(C(\theta, \dot{\theta}) \dot{\theta} + g(\theta)\right) - \Lambda(\theta) \dot{J}_b(\theta) J_b^{-1}(\theta) V_b
\end{align}

由于 $\Lambda(\theta) = J_b^{-T}(\theta) M(\theta) J_b^{-1}(\theta)$,代入得:
\begin{align}
\tilde{\eta}(\theta, V_b) &= J_b^{-T}(\theta) \left(C(\theta, \dot{\theta}) \dot{\theta} + g(\theta)\right) - J_b^{-T}(\theta) M(\theta) J_b^{-1}(\theta) \dot{J}_b(\theta) J_b^{-1}(\theta) V_b \\
&= J_b^{-T}(\theta) \left(C(\theta, \dot{\theta}) \dot{\theta} + g(\theta) - M(\theta) J_b^{-1}(\theta) \dot{J}_b(\theta) V_b\right)
\end{align}

因此两种形式等价。

\subsection{计算步骤(代码实现)}

代码中的计算步骤:
\begin{enumerate}
    \item 计算雅可比矩阵:$J_b(\theta)$
    \item 计算雅可比逆:$J_b^{-1}(\theta)$(使用伪逆或直接求逆)
    \item 计算关节速度:$\dot{\theta} = J_b^{-1}(\theta) V_b$
    \item 计算$h(\theta, \dot{\theta}) = C(\theta, \dot{\theta}) \dot{\theta} + g(\theta)$
    \item 计算雅可比时间导数:$\dot{J}_b(\theta)$
    \item 计算任务空间质量矩阵:$\Lambda(\theta)$
    \item 第一项:$J_b^{-T}(\theta) h(\theta, \dot{\theta})$
    \item 第二项:$\Lambda(\theta) \dot{J}_b(\theta) J_b^{-1}(\theta) V_b$
    \item 组合:$\tilde{\eta} = \text{第一项} - \text{第二项}$
\end{enumerate}

\subsection{简化形式(忽略雅可比时间导数项)}

在某些应用中,可以忽略雅可比时间导数项,得到简化形式:
\begin{equation}
\tilde{\eta}(\theta, V_b) \approx J_b^{-T}(\theta) \left(C(\theta, \dot{\theta}) \dot{\theta} + g(\theta)\right)
\end{equation}

但代码中使用了完整形式(公式8.91),包含雅可比时间导数项,以获得更精确的动力学补偿。

\section{完整的控制律 $\tau$(公式11.61)}

混合运动-力控制律为:
\begin{equation}
\begin{split}
\tau = J_b^T(\theta)\Bigg[&P(\theta)\left(\tilde{\Lambda}(\theta)\frac{d}{dt}([Ad_{X^{-1}X_d}]V_d) + K_p X_e + K_i\int_0^t X_e(\tau)d\tau + K_d V_e\right) \\
&+ (I - P(\theta))\left(F_d + K_{fp}F_e + K_{fi}\int_0^t F_e(\tau)d\tau\right) + \tilde{\eta}(\theta, V_b)\Bigg]
\end{split}
\end{equation}

其中各项说明:

\subsection{运动控制部分}
\begin{equation}
F_{\text{motion}} = P(\theta)\left(\tilde{\Lambda}(\theta)\frac{d}{dt}([Ad_{X^{-1}X_d}]V_d) + K_p X_e + K_i\int_0^t X_e(\tau)d\tau + K_d V_e\right)
\end{equation}

包含:
\begin{itemize}
    \item 前馈加速度项:$\tilde{\Lambda}(\theta)\frac{d}{dt}([Ad_{X^{-1}X_d}]V_d)$
    \item 比例项:$K_p X_e$($K_p \in \mathbb{R}^{6 \times 6}$)
    \item 积分项:$K_i\int_0^t X_e(\tau)d\tau$($K_i \in \mathbb{R}^{6 \times 6}$)
    \item 微分项:$K_d V_e$($K_d \in \mathbb{R}^{6 \times 6}$)
\end{itemize}

\subsection{力控制部分}
\begin{equation}
F_{\text{force}} = (I - P(\theta))\left(F_d + K_{fp}F_e + K_{fi}\int_0^t F_e(\tau)d\tau\right)
\end{equation}

其中:
\begin{itemize}
    \item $F_d = [0, 0, -15, 0, 0, 0]^T$:期望wrench(仅Z方向法向力)
    \item $F_e = F_d - F_{\text{curr}}$:力误差($F_{\text{curr}}$是当前测量的wrench)
    \item $K_{fp} \in \mathbb{R}^{6 \times 6}$:力比例增益矩阵
    \item $K_{fi} \in \mathbb{R}^{6 \times 6}$:力积分增益矩阵
\end{itemize}

\subsection{动力学补偿}
\begin{equation}
\tilde{\eta}(\theta, V_b) = J_b^{-T}(\theta) \left(C(\theta, \dot{\theta}) \dot{\theta} + g(\theta)\right)
\end{equation}

补偿任务空间中的Coriolis和重力效应。

\section{增益矩阵设计}

\subsection{运动控制增益}
\begin{equation}
K_p = \begin{bmatrix}
K_{p,\text{pos}} & 0 \\
0 & K_{p,\text{rot}}
\end{bmatrix}, \quad
K_i = \begin{bmatrix}
K_{i,\text{pos}} & 0 \\
0 & K_{i,\text{rot}}
\end{bmatrix}, \quad
K_d = \begin{bmatrix}
K_{d,\text{pos}} & 0 \\
0 & K_{d,\text{rot}}
\end{bmatrix}
\end{equation}

其中:
\begin{itemize}
    \item $K_{p,\text{pos}} = \text{diag}(1000, 1000, 500)$:位置增益
    \item $K_{p,\text{rot}} = \text{diag}(50, 50, 30)$:旋转增益
    \item $K_{i,\text{pos}} = \text{diag}(20, 20, 10)$:位置积分增益
    \item $K_{d,\text{pos}} = \text{diag}(80, 80, 40)$:位置微分增益
\end{itemize}

\subsection{力控制增益}
\begin{equation}
K_{fp} = \text{diag}(0.5, 0.5, 1.0, 10.0, 10.0, 5.0), \quad
K_{fi} = \text{diag}(0.1, 0.1, 0.2, 2.0, 2.0, 1.0)
\end{equation}

\section{控制律的物理意义}

\begin{itemize}
    \item \textbf{投影矩阵$P$}:将运动控制wrench投影到切向运动子空间(XY平面),不影响约束方向(Z方向和旋转)
    \item \textbf{投影矩阵$I-P$}:将力控制wrench投影到约束方向(Z方向和旋转),不影响切向运动
    \item \textbf{解耦性}:由于$P$和$I-P$正交,运动控制和力控制相互解耦,可以独立设计
    \item \textbf{稳定性}:每个子空间上的控制器继承各自子空间的误差动力学和稳定性分析
\end{itemize}

\end{document}

