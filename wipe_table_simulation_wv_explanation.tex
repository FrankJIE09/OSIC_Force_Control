\documentclass[12pt,a4paper]{article}
\usepackage[UTF8]{ctex}
\usepackage{amsmath,amssymb,amsthm}
\usepackage{graphicx}
\usepackage{geometry}
\usepackage{hyperref}
\usepackage{amsthm}
\usepackage{bm}
\usepackage{algorithm}
\usepackage{algorithmic}

\geometry{margin=2.5cm}

\title{圆形擦拭任务控制算法详解}
\subtitle{基于混合力位控制的空间坐标系实现}
\author{机器人控制理论}
\date{\today}

\newtheorem{definition}{定义}
\newtheorem{theorem}{定理}
\newtheorem{lemma}{引理}

\begin{document}

\maketitle

\section{引言}

本文档详细说明圆形擦拭任务的完整控制算法,包括:
\begin{itemize}
    \item 混合力位控制(Hybrid Force/Position Control)
    \item 空间坐标系下的动力学计算
    \item 投影矩阵方法
    \item 轨迹生成与状态机
    \item 力传感器数据处理
\end{itemize}

\section{系统概述}

\subsection{任务描述}

圆形擦拭任务分为三个阶段:
\begin{enumerate}
    \item \textbf{接近阶段(APPROACH)}:机器人移动到工作表面上方
    \item \textbf{下降阶段(DESCEND)}:缓慢下降直到接触表面
    \item \textbf{擦拭阶段(WIPE)}:在XY平面画圆,Z轴保持恒定压力
\end{enumerate}

\subsection{坐标系定义}

所有计算在\textbf{空间坐标系(Spatial Frame/World Frame)}中进行:
\begin{itemize}
    \item 原点:世界坐标系原点
    \item 姿态:末端执行器相对于世界坐标系的旋转
    \item 位置:末端执行器在世界坐标系中的位置
\end{itemize}

\section{动力学模型}

\subsection{关节空间动力学}

关节空间动力学方程为:
\begin{equation}
M(\theta)\ddot{\theta} + h(\theta, \dot{\theta}) = \tau
\end{equation}
其中:
\begin{itemize}
    \item $M(\theta) \in \mathbb{R}^{7 \times 7}$:关节空间质量矩阵
    \item $h(\theta, \dot{\theta}) \in \mathbb{R}^{7}$:科里奥利、向心力和重力项
    \item $\tau \in \mathbb{R}^{7}$:关节力矩
    \item $\theta, \dot{\theta}, \ddot{\theta} \in \mathbb{R}^{7}$:关节角度、速度、加速度
\end{itemize}

\subsection{任务空间动力学}

任务空间动力学方程为:
\begin{equation}
F_s = \Lambda_s(\theta) \dot{V}_s + \eta_s(\theta, V_s)
\end{equation}
其中:
\begin{itemize}
    \item $F_s \in \mathbb{R}^{6}$:空间坐标系中的wrench(力和力矩)
    \item $\Lambda_s(\theta) \in \mathbb{R}^{6 \times 6}$:任务空间质量矩阵
    \item $\eta_s(\theta, V_s) \in \mathbb{R}^{6}$:任务空间科里奥利项
    \item $V_s \in \mathbb{R}^{6}$:空间坐标系中的twist(速度和角速度)
\end{itemize}

\subsection{雅可比矩阵}

空间雅可比矩阵$J_s(\theta)$将关节速度映射到任务空间速度:
\begin{equation}
V_s = J_s(\theta) \dot{\theta}
\end{equation}

任务空间质量矩阵和科里奥利项的计算:
\begin{align}
\Lambda_s(\theta) &= (J_s(\theta) M^{-1}(\theta) J_s^T(\theta))^{-1} \\
\eta_s(\theta, V_s) &= \Lambda_s(\theta) [J_s(\theta) M^{-1}(\theta) h(\theta, \dot{\theta}) - \dot{J}_s(\theta) \dot{\theta}]
\end{align}

\section{混合力位控制}

\subsection{约束矩阵}

当圆盘与表面接触时,Z轴方向被约束。约束矩阵$A_s$定义为:
\begin{equation}
A_s = \begin{bmatrix} 0 & 0 & 0 & 0 & 0 & 1 \end{bmatrix}
\end{equation}
表示约束Z轴平移速度:$v_z = 0$。

约束方程:
\begin{equation}
A_s V_s = 0
\end{equation}

\subsection{投影矩阵}

投影矩阵$P_s$将任务空间wrench投影到运动子空间:
\begin{equation}
P_s = I - A_s^T (A_s \Lambda_s^{-1} A_s^T)^{-1} A_s \Lambda_s^{-1}
\end{equation}

\textbf{性质}:
\begin{itemize}
    \item $P_s$的秩为$n-k = 6-1 = 5$(运动子空间维度)
    \item $I-P_s$的秩为$k = 1$(力子空间维度)
    \item $P_s$和$I-P_s$正交:$P_s(I-P_s) = 0$
    \item $P_s$将Z轴方向的运动控制移除
    \item $I-P_s$将Z轴方向的力控制保留
\end{itemize}

\subsection{控制律}

完整的混合力位控制律为:
\begin{equation}
F_s = P_s F_{\text{motion}} + (I - P_s) F_{\text{force}} + \eta_s(\theta, V_s)
\end{equation}

\section{运动控制}

\subsection{配置误差}

空间配置误差定义为:
\begin{equation}
X_e = \begin{bmatrix} \omega_e \\ p_e \end{bmatrix}
\end{equation}
其中:
\begin{align}
\omega_e &= \log(R_d R_c^T) \quad \text{(旋转误差向量)} \\
p_e &= p_d - p_c \quad \text{(位置误差向量)}
\end{align}

其中$R_c, p_c$是当前位姿,$R_d, p_d$是期望位姿。

\subsection{速度误差}

速度误差定义为:
\begin{equation}
V_e = V_d - V_s
\end{equation}
在代码中,假设期望速度为0,因此:
\begin{equation}
V_e = -V_s
\end{equation}

\subsection{运动控制力}

运动控制力使用PD控制加积分项:
\begin{equation}
F_{\text{motion}} = P_s \Lambda_s (K_p X_e + K_d V_e + K_i \int_0^t X_e(\tau) d\tau)
\end{equation}

其中增益矩阵为:
\begin{align}
K_p &= \text{diag}([80, 80, 80, 400, 400, 400]) \quad \text{(旋转,位置)} \\
K_d &= \text{diag}([5, 5, 5, 20, 20, 20]) \\
K_i &= \text{diag}([0.1, 0.1, 0.1, 10, 10, 10])
\end{align}

\textbf{积分限幅(Anti-windup)}:
\begin{equation}
\int_0^t X_e(\tau) d\tau = \text{clip}\left(\int_0^t X_e(\tau) d\tau, -0.5, 0.5\right)
\end{equation}

\section{力控制}

\subsection{力误差}

力误差定义为:
\begin{equation}
F_e = F_d - F_{\text{curr}}
\end{equation}
其中:
\begin{itemize}
    \item $F_d = [0, 0, 0, 0, 0, -15]^T$:期望wrench(Z轴压力-15N)
    \item $F_{\text{curr}} = [0, 0, 0, 0, 0, f_z]^T$:当前测量的力
\end{itemize}

\subsection{力控制律}

力控制使用PI控制:
\begin{equation}
F_{\text{force}} = (I - P_s) \left(F_d + K_{fp} F_e + K_{fi} \int_0^t F_e(\tau) d\tau\right)
\end{equation}

其中增益矩阵为:
\begin{align}
K_{fp} &= \text{diag}([0, 0, 0, 0, 0, 0.5]) \\
K_{fi} &= \text{diag}([0, 0, 0, 0, 0, 2.0])
\end{align}

\textbf{力积分限幅}:
\begin{equation}
\int_0^t F_e(\tau) d\tau = \text{clip}\left(\int_0^t F_e(\tau) d\tau, -50, 50\right)
\end{equation}

\section{力传感器处理}

\subsection{接触力检测}

使用MuJoCo的touch传感器测量接触力:
\begin{equation}
f_{\text{raw}} = \text{touch}(\text{disk\_center\_site}) + \text{touch}(\text{disk\_bottom\_site})
\end{equation}

\subsection{低通滤波}

对原始力信号进行低通滤波以减少噪声:
\begin{equation}
f_{\text{filtered}}(t) = \alpha \cdot f_{\text{raw}}(t) + (1 - \alpha) \cdot f_{\text{filtered}}(t - \Delta t)
\end{equation}
其中$\alpha = 0.1$是滤波系数。

\subsection{接触判断}

接触判断条件:
\begin{equation}
\text{is\_contact} = \begin{cases}
\text{True} & \text{if } |f_{\text{filtered}}| > 0.5 \text{ N} \\
\text{False} & \text{otherwise}
\end{cases}
\end{equation}

\section{轨迹生成}

\subsection{阶段1:接近阶段($t < 2.0$秒)}

目标位置:
\begin{equation}
p_d(t) = \begin{bmatrix} 0.5 \\ 0.0 \\ 0.3 \end{bmatrix}
\end{equation}

目标姿态(末端垂直向下):
\begin{equation}
R_d = \text{Rot}(x, \pi) = \begin{bmatrix}
1 & 0 & 0 \\
0 & -1 & 0 \\
0 & 0 & -1
\end{bmatrix}
\end{equation}

\subsection{阶段2:下降阶段($2.0 \leq t < 4.0$秒)}

目标位置(线性插值):
\begin{equation}
p_d(t) = \begin{bmatrix} 0.5 \\ 0.0 \\ z_d(t) \end{bmatrix}
\end{equation}
其中:
\begin{equation}
z_d(t) = 0.3 \cdot (1 - \alpha) + 0.14 \cdot \alpha, \quad \alpha = \frac{t - 2.0}{2.0}
\end{equation}

\subsection{阶段3:擦拭阶段($t \geq 4.0$秒)}

目标位置(圆形轨迹):
\begin{equation}
p_d(t) = \begin{bmatrix}
x_c + r \cos(\omega t') \\
y_c + r \sin(\omega t') \\
0.14
\end{bmatrix}
\end{equation}
其中:
\begin{itemize}
    \item 圆心:$(x_c, y_c) = (0.5, 0.0)$
    \item 半径:$r = 0.1$ m
    \item 角频率:$\omega = 1.0$ rad/s
    \item 时间:$t' = t - 4.0$(从擦拭阶段开始的时间)
\end{itemize}

\section{控制算法流程}

\subsection{主控制循环}

\textbf{算法1:主控制循环}

\begin{algorithm}
\caption{圆形擦拭任务控制算法}
\begin{algorithmic}[1]
\REQUIRE 当前时间$t$,时间步长$\Delta t$
\ENSURE 关节力矩$\tau$
\STATE 1. 获取当前状态:$q, \dot{q}, p_c, R_c$
\STATE 2. 计算空间动力学:$J_s, \Lambda_s, \eta_s, V_s$
\STATE 3. 读取并滤波接触力:$f_{\text{filtered}}$
\STATE 4. 判断接触状态:$\text{is\_contact} = |f_{\text{filtered}}| > 0.5$
\STATE 5. 生成参考轨迹:$p_d, R_d, \text{mode}$
\STATE 6. 计算配置误差:$X_e = [\log(R_d R_c^T), p_d - p_c]^T$
\STATE 7. 计算速度误差:$V_e = -V_s$
\STATE 8. 更新积分项:$\int X_e \leftarrow \int X_e + X_e \Delta t$(限幅)
\STATE 9. 计算约束矩阵:$A_s = [0, 0, 0, 0, 0, 1]$(如果接触)
\STATE 10. 计算投影矩阵:$P_s = I - A_s^T (A_s \Lambda_s^{-1} A_s^T)^{-1} A_s \Lambda_s^{-1}$
\STATE 11. 计算运动控制力:$F_{\text{motion}} = P_s \Lambda_s (K_p X_e + K_d V_e + K_i \int X_e)$
\STATE 12. 计算力控制力:$F_{\text{force}} = (I-P_s)(F_d + K_{fp} F_e + K_{fi} \int F_e)$(如果接触)
\STATE 13. 合成总wrench:$F_s = F_{\text{motion}} + F_{\text{force}} + \eta_s$
\STATE 14. 映射到关节力矩:$\tau = J_s^T F_s$
\STATE 15. 添加零空间阻尼:$\tau \leftarrow \tau + (I - J_s^T (J_s^T)^\dagger) \tau_{\text{null}}$
\STATE 16. 力矩限幅:$\tau \leftarrow \text{clip}(\tau, -87, 87)$
\STATE 17. 应用控制:$\text{ctrl} \leftarrow \tau$
\end{algorithmic}
\end{algorithm}

\section{零空间控制}

\subsection{零空间阻尼}

为了稳定冗余自由度,添加零空间阻尼:
\begin{equation}
\tau_{\text{null}} = -K_{\text{null}} \dot{\theta}
\end{equation}
其中$K_{\text{null}} = 2.0$是阻尼系数。

零空间投影:
\begin{equation}
\tau_{\text{null\_projected}} = (I - J_s^T (J_s^T)^\dagger) \tau_{\text{null}}
\end{equation}

其中$(J_s^T)^\dagger$是$J_s^T$的伪逆。

\section{完整控制律总结}

\subsection{控制律公式}

完整的控制律为:
\begin{align}
F_s &= P_s \Lambda_s (K_p X_e + K_d V_e + K_i \int X_e) \nonumber \\
&\quad + (I - P_s) \left(F_d + K_{fp} F_e + K_{fi} \int F_e\right) + \eta_s \\
\tau &= J_s^T F_s + (I - J_s^T (J_s^T)^\dagger) (-K_{\text{null}} \dot{\theta}) \\
\tau &= \text{clip}(\tau, -\tau_{\max}, \tau_{\max})
\end{align}

\subsection{控制模式切换}

控制模式根据时间和接触状态切换:
\begin{equation}
\text{mode} = \begin{cases}
\text{APPROACH} & \text{if } t < 2.0 \\
\text{DESCEND} & \text{if } 2.0 \leq t < 4.0 \\
\text{WIPE} & \text{if } t \geq 4.0
\end{cases}
\end{equation}

力控制启用条件:
\begin{equation}
\text{enable\_force\_control} = \text{is\_contact} \land (\text{mode} \in \{\text{DESCEND}, \text{WIPE}\})
\end{equation}

\section{参数总结}

\subsection{控制增益}

\begin{table}[h]
\centering
\caption{控制增益参数}
\begin{tabular}{|l|l|l|}
\hline
\textbf{参数} & \textbf{值} & \textbf{说明} \\
\hline
$K_p$(旋转) & $\text{diag}([80, 80, 80])$ & 姿态比例增益 \\
\hline
$K_p$(位置) & $\text{diag}([400, 400, 400])$ & 位置比例增益 \\
\hline
$K_d$(旋转) & $\text{diag}([5, 5, 5])$ & 姿态微分增益 \\
\hline
$K_d$(位置) & $\text{diag}([20, 20, 20])$ & 位置微分增益 \\
\hline
$K_i$(旋转) & $\text{diag}([0.1, 0.1, 0.1])$ & 姿态积分增益 \\
\hline
$K_i$(位置) & $\text{diag}([10, 10, 10])$ & 位置积分增益 \\
\hline
$K_{fp}$(Z轴) & $0.5$ & 力控制比例增益 \\
\hline
$K_{fi}$(Z轴) & $2.0$ & 力控制积分增益 \\
\hline
$K_{\text{null}}$ & $2.0$ & 零空间阻尼系数 \\
\hline
\end{tabular}
\end{table}

\subsection{任务参数}

\begin{table}[h]
\centering
\caption{任务参数}
\begin{tabular}{|l|l|}
\hline
\textbf{参数} & \textbf{值} \\
\hline
期望压力 $F_d$ & $-15.0$ N \\
\hline
接触阈值 & $0.5$ N \\
\hline
滤波系数 $\alpha$ & $0.1$ \\
\hline
圆心 $(x_c, y_c)$ & $(0.5, 0.0)$ m \\
\hline
半径 $r$ & $0.1$ m \\
\hline
角频率 $\omega$ & $1.0$ rad/s \\
\hline
接近高度 & $0.3$ m \\
\hline
接触高度 & $0.14$ m \\
\hline
力矩限制 $\tau_{\max}$ & $87$ N·m(前4个关节),$12$ N·m(后3个关节) \\
\hline
\end{tabular}
\end{table}

\section{实现细节}

\subsection{积分限幅}

为了防止积分饱和,对积分项进行限幅:
\begin{align}
\int X_e &\in [-0.5, 0.5]^6 \\
\int F_e &\in [-50, 50]^6
\end{align}

\subsection{力矩限制}

关节力矩被限制在安全范围内:
\begin{equation}
\tau_i = \begin{cases}
-87 & \text{if } \tau_i < -87 \text{ and } i \leq 4 \\
87 & \text{if } \tau_i > 87 \text{ and } i \leq 4 \\
-12 & \text{if } \tau_i < -12 \text{ and } i > 4 \\
12 & \text{if } \tau_i > 12 \text{ and } i > 4 \\
\tau_i & \text{otherwise}
\end{cases}
\end{equation}

\section{结论}

本文档详细说明了圆形擦拭任务的完整控制算法:

\begin{enumerate}
    \item \textbf{混合力位控制}:使用投影矩阵分离运动和力控制子空间
    \item \textbf{空间坐标系}:所有计算在世界坐标系中进行
    \item \textbf{三阶段任务}:接近、下降、擦拭
    \item \textbf{力传感器处理}:低通滤波和接触检测
    \item \textbf{零空间控制}:稳定冗余自由度
\end{enumerate}

该控制算法能够实现:
\begin{itemize}
    \item 平滑接近工作表面
    \item 稳定接触并保持恒定压力
    \item 在XY平面精确跟踪圆形轨迹
    \item 在Z轴方向保持力控制
\end{itemize}

\end{document}

