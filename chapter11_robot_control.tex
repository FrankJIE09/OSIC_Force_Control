\documentclass[12pt,a4paper]{article}
\usepackage[UTF8]{ctex}
\usepackage{amsmath,amssymb,amsthm}
\usepackage{graphicx}
\usepackage{geometry}
\usepackage{hyperref}
\usepackage{algorithm}
\usepackage{algorithmic}
\usepackage{listings}
\usepackage{xcolor}

\geometry{margin=2.5cm}

\title{第11章\quad 机器人控制}
\author{现代机器人学:力学、规划与控制\\
Kevin M. Lynch 和 Frank C. Park\\
中文翻译}
\date{}

\begin{document}

\maketitle

\section{引言}

机器人臂可以根据任务和环境表现出多种不同的行为。它可以作为程序化运动的来源,用于诸如将物体从一个地方移动到另一个地方或为喷漆枪追踪轨迹等任务。它可以作为力的来源,例如将抛光轮应用于工件时。在诸如在黑板上写字等任务中,它必须在某些方向上控制力(力必须将粉笔压向黑板),而在其他方向上控制运动(运动必须在黑板平面内)。当机器人的目的是作为触觉显示器,渲染虚拟环境时,我们可能希望它表现得像弹簧、阻尼器或质量,响应施加在其上的力。

在每种情况下,机器人控制器的工作是将任务规范转换为执行器处的力和力矩。实现上述行为的控制策略被称为运动控制、力控制、混合运动-力控制或阻抗控制。这些行为中哪一种合适取决于任务和环境。例如,当末端执行器与某物接触时,力控制目标是有意义的,但当它在自由空间中移动时则不然。我们还有一个由力学施加的基本约束,与环境无关:机器人不能在同一方向上独立控制运动和力。如果机器人施加运动,则环境将确定力;如果机器人施加力,则环境将确定运动。

一旦我们选择了与任务和环境一致的控制目标,我们可以使用反馈控制来实现它。反馈控制使用位置、速度和力传感器来测量机器人的实际行为,将其与期望行为进行比较,并调制发送到执行器的控制信号。反馈在几乎所有机器人系统中都使用。

在本章中,我们重点关注:关节空间和任务空间中的运动控制反馈控制;力控制;混合运动-力控制;以及阻抗控制。

\section{控制系统概述}

典型的控制框图如图11.1(a)所示。传感器通常是:用于关节位置和角度感测的电位器、编码器或旋转变压器;用于关节速度感测的转速计;关节力-力矩传感器;和/或在臂末端和末端执行器之间的"手腕"处的多轴力-力矩传感器。控制器以数百到数千Hz的速率对传感器进行采样并更新其发送到执行器的控制信号。在大多数机器人应用中,高于此的控制更新速率益处有限,考虑到与机器人和环境动力学相关的时间常数。在我们的分析中,我们将忽略采样时间非零的事实,并将控制器视为在连续时间中实现。

虽然转速计可用于直接速度感测,但常见的方法是使用数字滤波器在连续时间步长上对位置信号进行数值差分。低通滤波器通常与差分滤波器结合使用,以减少由于差分位置信号的量化而产生的高频信号内容。

如第8.9节所述,有多种不同的技术用于产生机械功率、转换速度和力,并传输到机器人关节。在本章中,我们将每个关节的放大器、执行器和传动装置集中在一起,并将它们视为从低功率控制信号到力和力矩的变换器。这个假设,连同理想传感器的假设,允许我们将图11.1(a)的框图简化为图11.1(b)所示的框图,其中控制器直接产生力和力矩。本章的其余部分处理进入图11.1(b)中"控制器"框的控制算法。

真实的机器人系统会受到关节和连杆中的柔性和振动、齿轮和传动装置中的间隙、执行器饱和限制以及传感器有限分辨率的影响。这些在设计和控制中提出了重要问题,但它们超出了本章的范围。

\begin{figure}[h]
\centering
% 图11.1的占位符
\caption{图11.1:(a)典型的机器人控制系统。内部控制回路用于帮助放大器和执行器实现期望的力或力矩。例如,处于力矩控制模式的直流电机放大器可以感测实际流过电机的电流,并实现局部控制器以更好地匹配期望电流,因为电流与电机产生的力矩成正比。或者,电机控制器可以通过在电机输出齿轮上使用应变计直接感测力矩,并使用该反馈闭合局部力矩控制回路。(b)具有理想传感器和直接产生力和力矩的控制器块的简化模型。这假设了部分(a)中放大器和执行器块的理想行为。未显示的是可以在动力学块之前注入的干扰力,或在动力学块之后注入的干扰力或运动。}
\end{figure}

\section{误差动力学}

在本节中,我们关注单个关节的受控动力学,因为这些概念很容易推广到多关节机器人的情况。

如果期望的关节位置是$\theta_d(t)$,实际关节位置是$\theta(t)$,则我们将关节误差定义为
\begin{equation}
\theta_e(t) = \theta_d(t) - \theta(t).
\end{equation}

控制系统的关节误差$\theta_e(t)$演化的微分方程称为误差动力学。反馈控制器的目的是创建误差动力学,使得$\theta_e(t)$随着$t$的增加而趋于零或小值。

\subsection{误差响应}

测试控制器工作效果的一种常见方法是指定非零初始误差$\theta_e(0)$,并查看控制器减少初始误差的速度和完整程度。我们将(单位)误差响应定义为受控系统在初始条件$\theta_e(0) = 1$和$\dot{\theta}_e(0) = \ddot{\theta}_e(0) = \cdots = 0$下的响应$\theta_e(t), t > 0$。

理想控制器会立即将误差驱动到零,并在所有时间保持误差为零。实际上,减少误差需要时间,误差可能永远不会完全消除。如图11.2所示,典型的误差响应$\theta_e(t)$可以用瞬态响应和稳态响应来描述。稳态响应的特征是稳态误差$e_{ss}$,它是$t \to \infty$时的渐近误差$\theta_e(t)$。瞬态响应的特征是超调和(2\%)稳定时间。2\%稳定时间是第一个时间$T$,使得对于所有$t \geq T$有$|\theta_e(t) - e_{ss}| \leq 0.02(\theta_e(0) - e_{ss})$(参见一对长虚线)。如果误差响应最初超过最终稳态误差,则发生超调,在这种情况下,超调定义为
\begin{equation}
\text{超调} = \frac{\theta_{e,\min} - e_{ss}}{\theta_e(0) - e_{ss}} \times 100\%,
\end{equation}
其中$\theta_{e,\min}$是误差达到的最小正值。

良好的误差响应的特征是:
\begin{itemize}
\item 很少或没有稳态误差,
\item 很少或没有超调,以及
\item 短的2\%稳定时间。
\end{itemize}

\begin{figure}[h]
\centering
% 图11.2的占位符
\caption{图11.2:显示稳态误差$e_{ss}$、超调和2\%稳定时间的误差响应示例。}
\end{figure}

\subsection{线性误差动力学}

在本章中,我们主要处理线性系统,其误差动力学由以下形式的线性常微分方程描述:
\begin{equation}
a_p \theta_e^{(p)} + a_{p-1} \theta_e^{(p-1)} + \cdots + a_2 \ddot{\theta}_e + a_1 \dot{\theta}_e + a_0 \theta_e = c.
\end{equation}

这是一个$p$阶微分方程,因为存在$\theta_e$的$p$个时间导数。如果常数$c$为零,则微分方程(11.1)是齐次的;如果$c \neq 0$,则是非齐次的。

对于齐次($c = 0$)线性误差动力学,$p$阶微分方程(11.1)可以重写为
\begin{equation}
\theta_e^{(p)} = -\frac{1}{a_p}(a_{p-1} \theta_e^{(p-1)} + \cdots + a_2 \ddot{\theta}_e + a_1 \dot{\theta}_e + a_0 \theta_e) = -a'_0 \theta_e^{(p-1)} - \cdots - a'_2 \ddot{\theta}_e - a'_1 \dot{\theta}_e - a'_0 \theta_e.
\end{equation}

这个$p$阶微分方程可以通过定义向量$\mathbf{x} = (x_1, \ldots, x_p)$表示为$p$个耦合的一阶微分方程,其中
\begin{align}
x_1 &= \theta_e, \\
x_2 &= \dot{x}_1 = \dot{\theta}_e, \\
&\vdots \\
x_p &= \dot{x}_{p-1} = \theta_e^{(p-1)},
\end{align}
并将方程(11.2)写为
\begin{equation}
\dot{x}_p = -a'_0 x_1 - a'_1 x_2 - \cdots - a'_{p-1} x_p.
\end{equation}

然后$\dot{\mathbf{x}}(t) = A\mathbf{x}(t)$,其中$A$是$p \times p$矩阵。

通过与标量一阶微分方程$\dot{x}(t) = ax(t)$的类比,其解为$x(t) = e^{at}x(0)$,向量微分方程$\dot{\mathbf{x}}(t) = A\mathbf{x}(t)$的解为$\mathbf{x}(t) = e^{At}\mathbf{x}(0)$,使用矩阵指数,如我们在第3.2.3.1节中看到的。也类似于标量微分方程,如果$a$为负,其解从任何初始条件收敛到平衡点$x = 0$,如果矩阵$A$是负定的,即$A$的所有特征值(可能是复数)具有负实部,则微分方程$\dot{\mathbf{x}}(t) = A\mathbf{x}(t)$收敛到$\mathbf{x} = 0$。

$A$的特征值由$A$的特征多项式的根给出,即满足以下条件的复数值$s$:
\begin{equation}
\det(sI - A) = s^p + a'_{p-1} s^{p-1} + \cdots + a'_2 s^2 + a'_1 s + a'_0 = 0.
\end{equation}

方程(11.3)也是与$p$阶微分方程(11.1)相关的特征方程。

方程(11.3)的每个根具有负实部的必要条件是所有系数$a'_0, \ldots, a'_{p-1}$必须为正。对于$p = 1$和$2$,这个条件也是充分的。对于$p = 3$,条件$a'_2 a'_1 > a'_0$也必须成立。对于更高阶系统,必须满足其他条件。

如果方程(11.3)的每个根具有负实部,我们称误差动力学是稳定的。如果任何根具有正实部,误差动力学是不稳定的,误差$\|\theta_e(t)\|$可以随着$t \to \infty$无界增长。

对于二阶误差动力学,要记住的一个良好的机械类比是线性质量-弹簧-阻尼器(图11.3)。质量$m$的位置是$\theta_e$,外部力$f$施加到质量上。阻尼器对质量施加力$-b\dot{\theta}_e$,其中$b$是阻尼常数,弹簧对质量施加力$-k\theta_e$,其中$k$是弹簧常数。因此,质量的运动方程可以写为
\begin{equation}
m\ddot{\theta}_e + b\dot{\theta}_e + k\theta_e = f.
\end{equation}

\begin{figure}[h]
\centering
% 图11.3的占位符
\caption{图11.3:线性质量-弹簧-阻尼器。}
\end{figure}

在质量$m$接近零的极限情况下,二阶动力学(11.4)简化为一阶动力学
\begin{equation}
b\dot{\theta}_e + k\theta_e = f.
\end{equation}

通过一阶动力学,外力产生速度而不是加速度。

在以下小节中,我们考虑齐次情况($f = 0$)的一阶和二阶误差响应,其中$b, k > 0$,确保误差动力学稳定且误差收敛到零($e_{ss} = 0$)。

\subsubsection{一阶误差动力学}

$f = 0$的一阶误差动力学(11.5)可以写成形式
\begin{equation}
\dot{\theta}_e(t) + \frac{k}{b}\theta_e(t) = 0 \quad \text{或} \quad \dot{\theta}_e(t) + \frac{1}{\tau}\theta_e(t) = 0,
\end{equation}
其中$\tau = b/k$称为一阶微分方程的时间常数。微分方程(11.6)的解是
\begin{equation}
\theta_e(t) = e^{-t/\tau}\theta_e(0).
\end{equation}

时间常数$\tau$是一阶指数衰减衰减到其初始值约37\%的时间。误差响应$\theta_e(t)$由初始条件$\theta_e(0) = 1$定义。不同时间常数的误差响应图如图11.4所示。稳态误差为零,衰减指数误差响应中没有超调,2\%稳定时间通过求解
\begin{equation}
\frac{\theta_e(t)}{\theta_e(0)} = 0.02 = e^{-t/\tau}
\end{equation}
得到$t$来确定。求解,我们得到
\begin{equation}
\ln 0.02 = -t/\tau \to t = 3.91\tau,
\end{equation}
2\%稳定时间约为$4\tau$。随着弹簧常数$k$增加或阻尼常数$b$减少,响应变得更快。

\begin{figure}[h]
\centering
% 图11.4的占位符
\caption{图11.4:三个不同时间常数$\tau$的一阶误差响应。}
\end{figure}

\subsubsection{二阶误差动力学}

二阶误差动力学
\begin{equation}
\ddot{\theta}_e(t) + \frac{b}{m}\dot{\theta}_e(t) + \frac{k}{m}\theta_e(t) = 0
\end{equation}
可以写成标准二阶形式
\begin{equation}
\ddot{\theta}_e(t) + 2\zeta\omega_n \dot{\theta}_e(t) + \omega_n^2 \theta_e(t) = 0,
\end{equation}
其中$\omega_n$称为自然频率,$\zeta$称为阻尼比。对于质量-弹簧-阻尼器,$\omega_n = \sqrt{k/m}$和$\zeta = b/(2\sqrt{km})$。特征多项式
\begin{equation}
s^2 + 2\zeta\omega_n s + \omega_n^2 = 0
\end{equation}
的两个根是
\begin{equation}
s_1 = -\zeta\omega_n + \omega_n\sqrt{\zeta^2 - 1} \quad \text{和} \quad s_2 = -\zeta\omega_n - \omega_n\sqrt{\zeta^2 - 1}.
\end{equation}

二阶误差动力学(11.8)稳定当且仅当$\zeta\omega_n > 0$和$\omega_n^2 > 0$。

如果误差动力学稳定,则根据根$s_{1,2}$是实且不等($\zeta > 1$)、实且相等($\zeta = 1$)还是复共轭($\zeta < 1$),微分方程有三种类型的解$\theta_e(t)$。

\begin{itemize}
\item \textbf{过阻尼:} $\zeta > 1$。根$s_{1,2}$是实且不同的,微分方程(11.8)的解是
\begin{equation}
\theta_e(t) = c_1 e^{s_1 t} + c_2 e^{s_2 t},
\end{equation}
其中$c_1$和$c_2$可以从初始条件计算。响应是两个衰减指数的和,时间常数为$\tau_1 = -1/s_1$和$\tau_2 = -1/s_2$。解中"较慢"的时间常数由较不负的根给出,$s_1 = -\zeta\omega_n + \omega_n\sqrt{\zeta^2 - 1}$。

(单位)误差响应的初始条件是$\theta_e(0) = 1$和$\dot{\theta}_e(0) = 0$,常数$c_1$和$c_2$可以计算为
\begin{equation}
c_1 = \frac{1}{2} + \frac{\zeta}{2\sqrt{\zeta^2 - 1}} \quad \text{和} \quad c_2 = \frac{1}{2} - \frac{\zeta}{2\sqrt{\zeta^2 - 1}}.
\end{equation}

\item \textbf{临界阻尼:} $\zeta = 1$。根$s_{1,2}$是实且相等的,$s_1 = s_2 = -\omega_n$。解是
\begin{equation}
\theta_e(t) = (c_1 + c_2 t)e^{-\omega_n t},
\end{equation}
其中$c_1$和$c_2$可以从初始条件计算。对于单位误差响应,$c_1 = 1$和$c_2 = \omega_n$。

\item \textbf{欠阻尼:} $\zeta < 1$。根$s_{1,2}$是复共轭的。解是
\begin{equation}
\theta_e(t) = e^{-\zeta\omega_n t}(c_1 \cos(\omega_d t) + c_2 \sin(\omega_d t)),
\end{equation}
其中$\omega_d = \omega_n\sqrt{1 - \zeta^2}$是阻尼自然频率。对于单位误差响应,$c_1 = 1$和$c_2 = \zeta\omega_n/\omega_d$。
\end{itemize}

\section{速度输入的运动控制}

当机器人控制器直接控制关节速度时,我们使用速度输入的运动控制。这种控制方法适用于具有速度控制执行器的机器人系统。

\subsection{关节空间控制}

在关节空间中,我们控制每个关节的速度$\dot{\theta}_i$以跟踪期望的关节轨迹$\theta_{d,i}(t)$。基本的控制律是
\begin{equation}
\dot{\theta} = \dot{\theta}_d + K_p(\theta_d - \theta),
\end{equation}
其中$K_p$是正定增益矩阵。这种控制方法简单有效,但需要精确的关节位置反馈。

\subsection{任务空间控制}

在任务空间中,我们控制末端执行器的速度以跟踪期望的末端执行器轨迹。控制律需要考虑雅可比矩阵:
\begin{equation}
\dot{\theta} = J^{-1}(\theta)\left(\dot{x}_d + K_p(x_d - x)\right),
\end{equation}
其中$J(\theta)$是雅可比矩阵,$x_d$是期望的末端执行器位置,$x$是实际位置。

\section{力矩或力输入的运动控制}

步进电机控制的机器人通常仅限于低或可预测的力-力矩要求的应用。此外,机器人控制工程师不依赖现成的电机放大器的速度控制模式,因为这些速度控制算法不使用机器人的动力学模型。相反,机器人控制工程师在力矩控制模式下使用放大器:放大器的输入是期望的力矩(或力)。这允许机器人控制工程师在控制律设计中使用机器人的动力学模型。

在本节中,控制器产生关节力矩和力以尝试在关节空间或任务空间中跟踪期望轨迹。同样,主要思想可以通过具有单个关节的机器人很好地说明,因此我们从那里开始,然后推广到多关节机器人。

\subsection{单个关节的运动控制}

考虑连接到单个连杆的单个电机,如图11.11所示。设$\tau$为电机的力矩,$\theta$为连杆的角度。动力学可以写为
\begin{equation}
\tau = M\ddot{\theta} + mgr\cos\theta,
\end{equation}
其中$M$是连杆关于旋转轴的标量惯性,$m$是连杆的质量,$r$是从轴到连杆质心的距离,$g \geq 0$是重力加速度。

根据模型(11.19),没有耗散:如果使连杆运动然后将$\tau$设置为零,连杆将永远运动。当然,这是不现实的;在各种轴承、齿轮和传动装置中必然存在摩擦。摩擦建模是一个活跃的研究领域,但在许多情况下,简单的线性摩擦模型就足够了。

在简单模型中,旋转摩擦是由于粘性摩擦力,因此
\begin{equation}
\tau_{\text{fric}} = b\dot{\theta},
\end{equation}
其中$b > 0$。加入摩擦力矩,我们的最终模型是
\begin{equation}
\tau = M\ddot{\theta} + mgr\cos\theta + b\dot{\theta},
\end{equation}
可以更紧凑地写成
\begin{equation}
\tau = M\ddot{\theta} + h(\theta, \dot{\theta}),
\end{equation}
其中$h$包含所有仅依赖于状态而不依赖于加速度的项。

\subsubsection{反馈控制:PID控制}

常见的反馈控制器是线性比例-积分-微分控制,或PID控制。PID控制器是PI控制器(方程(11.13))加上一个与误差时间导数成比例的项:
\begin{equation}
\tau = K_p\theta_e + K_i\int_0^t\theta_e(t)dt + K_d\dot{\theta}_e,
\end{equation}
其中控制增益$K_p$、$K_i$和$K_d$为正。比例增益$K_p$充当虚拟弹簧,试图减小位置误差$\theta_e = \theta_d - \theta$。微分增益$K_d$充当虚拟阻尼器,试图减小速度误差$\dot{\theta}_e = \dot{\theta}_d - \dot{\theta}$。积分增益可用于减小或消除稳态误差。

\textbf{PD控制和二阶误差动力学} 现在考虑$K_i = 0$的情况。这称为PD控制。假设机器人在水平面内运动($g = 0$)。将PD控制律代入动力学(11.21),我们得到
\begin{equation}
M\ddot{\theta} + b\dot{\theta} = K_p(\theta_d - \theta) + K_d(\dot{\theta}_d - \dot{\theta}).
\end{equation}

如果控制目标是设定点控制,在常数$\theta_d$处,$\dot{\theta}_d = \ddot{\theta}_d = 0$,则$\theta_e = \theta_d - \theta$,$\dot{\theta}_e = -\dot{\theta}$,$\ddot{\theta}_e = -\ddot{\theta}$。方程(11.24)可以重写为
\begin{equation}
M\ddot{\theta}_e + (b + K_d)\dot{\theta}_e + K_p\theta_e = 0,
\end{equation}
或者,以标准二阶形式(11.8)写成
\begin{equation}
\ddot{\theta}_e + \frac{b + K_d}{M}\dot{\theta}_e + \frac{K_p}{M}\theta_e = 0 \quad \rightarrow \quad \ddot{\theta}_e + 2\zeta\omega_n\dot{\theta}_e + \omega_n^2\theta_e = 0,
\end{equation}
其中阻尼比$\zeta$和自然频率$\omega_n$为
\begin{equation}
\zeta = \frac{b + K_d}{2\sqrt{K_p M}} \quad \text{和} \quad \omega_n = \sqrt{\frac{K_p}{M}}.
\end{equation}

为了稳定性,$b + K_d$和$K_p$必须为正。如果误差动力学方程稳定,则稳态误差为零。为了无超调和快速响应,增益$K_d$和$K_p$应选择满足临界阻尼($\zeta = 1$)。

\textbf{PID控制和三阶误差动力学} 现在考虑设定点控制的情况,其中连杆在垂直平面内运动($g > 0$)。使用上述PD控制律,误差动力学现在可以写成
\begin{equation}
M\ddot{\theta}_e + (b + K_d)\dot{\theta}_e + K_p\theta_e = mgr\cos\theta.
\end{equation}

这意味着关节在满足$K_p\theta_e = mgr\cos\theta$的配置$\theta$处静止,即当$\theta_d \neq \pm\pi/2$时,最终误差$\theta_e$非零。原因是机器人必须提供非零力矩以在$\theta \neq \pm\pi/2$处保持连杆静止,但PD控制律只有在$\theta_e \neq 0$时才能在静止时产生非零力矩。我们可以通过增大增益$K_p$来使这个稳态误差变小,但如上所述,存在实际限制。

为了消除稳态误差,我们通过设置$K_i > 0$返回到PID控制器。这允许在位置误差为零时仍有非零稳态力矩;只有积分误差必须非零。

为了说明这是如何工作的,写下设定点误差动力学
\begin{equation}
M\ddot{\theta}_e + (b + K_d)\dot{\theta}_e + K_p\theta_e + K_i\int_0^t\theta_e(t)dt = \tau_{\text{dist}},
\end{equation}
其中$\tau_{\text{dist}}$是替代重力项$mgr\cos\theta$的干扰力矩。对两边求导,我们得到三阶误差动力学
\begin{equation}
M\theta_e^{(3)} + (b + K_d)\ddot{\theta}_e + K_p\dot{\theta}_e + K_i\theta_e = \dot{\tau}_{\text{dist}}.
\end{equation}

如果$\tau_{\text{dist}}$是常数,则方程(11.29)的右边为零,其特征方程为
\begin{equation}
s^3 + \frac{b + K_d}{M}s^2 + \frac{K_p}{M}s + \frac{K_i}{M} = 0.
\end{equation}

如果方程(11.30)的所有根都具有负实部,则误差动力学稳定,$\theta_e$收敛到零。

为了使方程(11.30)的所有根都具有负实部,必须满足以下控制增益的稳定性条件:
\begin{align}
K_d &> -b, \\
K_p &> 0, \\
\frac{(b + K_d)K_p}{M} &> K_i > 0.
\end{align}

因此,新增益$K_i$必须同时满足下界和上界。一个合理的设计策略是选择$K_p$和$K_d$以获得良好的瞬态响应,然后选择$K_i$足够大以有助于减小或消除稳态误差,但足够小以不会显著影响稳定性。

在实践中,许多机器人控制器中$K_i = 0$,因为稳定性是最重要的。可以采用其他技术来限制积分控制的不利稳定性影响,例如积分器抗饱和,它对误差积分允许增长的大小设置限制。

\subsubsection{前馈控制}

轨迹跟踪的另一种策略是使用机器人动力学模型来主动生成力矩,而不是等待误差。设控制器的动力学模型为
\begin{equation}
\tau = \tilde{M}(\theta)\ddot{\theta} + \tilde{h}(\theta, \dot{\theta}),
\end{equation}
其中如果$\tilde{M}(\theta) = M(\theta)$和$\tilde{h}(\theta, \dot{\theta}) = h(\theta, \dot{\theta})$,则模型是完美的。

给定来自轨迹生成器的$\theta_d$、$\dot{\theta}_d$和$\ddot{\theta}_d$,前馈力矩计算为
\begin{equation}
\tau(t) = \tilde{M}(\theta_d(t))\ddot{\theta}_d(t) + \tilde{h}(\theta_d(t), \dot{\theta}_d(t)).
\end{equation}

如果机器人动力学模型是精确的,并且没有初始状态误差,则机器人精确地跟踪期望轨迹。

因为总是存在建模误差,前馈控制总是与反馈结合使用,如下所述。

\subsubsection{前馈加反馈线性化}

所有实际控制器都使用反馈,因为机器人和环境动力学的模型都不会是完美的。尽管如此,一个好的模型可以用来改善性能并简化分析。

让我们将PID控制与机器人动力学模型$\{\tilde{M}, \tilde{h}\}$结合,以实现误差动力学
\begin{equation}
\ddot{\theta}_e + K_d\dot{\theta}_e + K_p\theta_e + K_i\int_0^t\theta_e(t)dt = c
\end{equation}
沿着任意轨迹,而不仅仅是到设定点。误差动力学(11.33)和PID增益的适当选择确保轨迹误差的指数衰减。

由于$\ddot{\theta}_e = \ddot{\theta}_d - \ddot{\theta}$,为了实现误差动力学(11.33),我们选择机器人的指令加速度为
\begin{equation}
\ddot{\theta} = \ddot{\theta}_d - \ddot{\theta}_e,
\end{equation}
然后将其与方程(11.33)结合得到
\begin{equation}
\ddot{\theta} = \ddot{\theta}_d + K_d\dot{\theta}_e + K_p\theta_e + K_i\int_0^t\theta_e(t)dt.
\end{equation}

将方程(11.34)中的$\ddot{\theta}$代入机器人动力学模型$\{\tilde{M}, \tilde{h}\}$,我们得到前馈加反馈线性化控制器,也称为逆动力学控制器或计算力矩控制器:
\begin{equation}
\tau = \tilde{M}(\theta)\left(\ddot{\theta}_d + K_p\theta_e + K_i\int_0^t\theta_e(t)dt + K_d\dot{\theta}_e\right) + \tilde{h}(\theta, \dot{\theta}).
\end{equation}

这个控制器包括由于使用计划加速度$\ddot{\theta}_d$而产生的前馈分量,并且被称为反馈线性化,因为$\theta$和$\dot{\theta}$的反馈用于生成线性误差动力学。$\tilde{h}(\theta, \dot{\theta})$项抵消了非线性依赖于状态的动力学,惯性模型$\tilde{M}(\theta)$将期望的关节加速度转换为关节力矩,实现简单的线性误差动力学。

\subsection{多关节机器人的运动控制}

上述应用于单关节机器人的方法可以直接推广到$n$关节机器人。区别在于动力学(11.22)现在采用更一般的向量值形式
\begin{equation}
\tau = M(\theta)\ddot{\theta} + h(\theta, \dot{\theta}),
\end{equation}
其中$n \times n$正定质量矩阵$M$现在是配置$\theta$的函数。通常,动力学(11.36)的分量是耦合的——一个关节的加速度是其他关节的位置、速度和力矩的函数。

我们区分两种类型的多关节机器人控制:\textbf{分散式控制},其中每个关节单独控制,关节之间不共享信息;和\textbf{集中式控制},其中所有$n$个关节的完整状态信息可用于计算每个关节的控制。

\subsubsection{分散式多关节控制}

控制多关节机器人的最简单方法是在每个关节上应用独立控制器,例如第11.4.1节讨论的单关节控制器。当动力学解耦时(至少近似地),分散式控制是合适的。当每个关节的加速度仅依赖于该关节的力矩、位置和速度时,动力学是解耦的。这要求质量矩阵是对角的,如笛卡尔或龙门机器人,其中前三个轴是正交的移动轴。这种机器人等价于三个单关节系统。

在没有重力的情况下,高度齿轮传动的机器人也实现了近似解耦。质量矩阵$M(\theta)$几乎是对角的,因为它主要由电机本身的表观惯性主导(见第8.9.2节)。各个关节的显著摩擦也有助于动力学的解耦。

\subsubsection{集中式多关节控制}

当重力和力矩显著且耦合时,或者当质量矩阵$M(\theta)$不能很好地用对角矩阵近似时,分散式控制可能无法产生可接受的性能。在这种情况下,图11.18的计算力矩控制器(11.35)可以推广到多关节机器人。配置$\theta$和$\theta_d$以及误差$\theta_e = \theta_d - \theta$现在是$n$维向量,正标量增益变为正定矩阵$K_p$、$K_i$、$K_d$:
\begin{equation}
\tau = \tilde{M}(\theta)\left(\ddot{\theta}_d + K_p\theta_e + K_i\int_0^t\theta_e(t)dt + K_d\dot{\theta}_e\right) + \tilde{h}(\theta, \dot{\theta}).
\end{equation}

通常,我们选择增益矩阵为$k_p I$、$k_i I$和$k_d I$,其中$k_p$、$k_i$和$k_d$是非负标量。通常,$k_i$选择为零。在$\tilde{M}$和$\tilde{h}$的精确动力学模型的情况下,每个关节的误差动力学简化为线性动力学(11.33)。

实现控制律(11.37)需要计算可能复杂的动力学。我们可能没有这些动力学的好模型,或者方程在伺服速率下计算可能过于昂贵。在这种情况下,如果期望速度和加速度很小,可以使用仅PID控制和重力补偿来获得(11.37)的近似:
\begin{equation}
\tau = K_p\theta_e + K_i\int_0^t\theta_e(t)dt + K_d\dot{\theta}_e + \tilde{g}(\theta).
\end{equation}

在零摩擦、完美重力补偿和PD设定点控制($K_i = 0$且$\dot{\theta}_d = \ddot{\theta}_d = 0$)的情况下,受控动力学可以写成
\begin{equation}
M(\theta)\ddot{\theta} + C(\theta, \dot{\theta})\dot{\theta} = K_p\theta_e - K_d\dot{\theta},
\end{equation}
其中科里奥利和向心项写成$C(\theta, \dot{\theta})\dot{\theta}$。我们现在可以定义一个虚拟"误差能量",它是存储在虚拟弹簧$K_p$中的"误差势能"和"误差动能"的总和:
\begin{equation}
V(\theta_e, \dot{\theta}_e) = \frac{1}{2}\theta_e^T K_p\theta_e + \frac{1}{2}\dot{\theta}_e^T M(\theta)\dot{\theta}_e.
\end{equation}

由于$\dot{\theta}_d = 0$,这简化为
\begin{equation}
V(\theta_e, \dot{\theta}) = \frac{1}{2}\theta_e^T K_p\theta_e + \frac{1}{2}\dot{\theta}^T M(\theta)\dot{\theta}.
\end{equation}

取时间导数并代入(11.39),我们得到
\begin{equation}
\dot{V} = -\dot{\theta}^T K_d\dot{\theta} \leq 0.
\end{equation}

这表明当$\dot{\theta} \neq 0$时误差能量在减小。如果$\dot{\theta} = 0$且$\theta \neq \theta_d$,虚拟弹簧确保$\ddot{\theta} \neq 0$,因此$\dot{\theta}_e$将再次变为非零,更多的误差能量将被耗散。因此,根据Krasovskii-LaSalle不变性原理(练习11.12),总误差能量单调减小,机器人从任何初始状态收敛到$\theta_d$($\theta_e = 0$)处的静止。

\subsection{任务空间的运动控制}

在第11.4.2节中,我们专注于关节空间中的运动控制。一方面,这很方便,因为关节限制在这个空间中很容易表达,机器人应该能够执行任何尊重这些限制的关节空间路径。轨迹自然由关节变量描述,没有奇异性或冗余性问题。

另一方面,由于机器人与外部环境及其中的物体交互,将运动表示为任务空间中末端执行器的轨迹可能更方便。设末端执行器轨迹由$(X(t), V_b(t))$指定,其中$X \in SE(3)$且$[V_b] = X^{-1}\dot{X}$,即twist $V_b$在末端执行器坐标系$\{b\}$中表示。假设关节空间中相应的轨迹是可行的,我们现在有两个控制选项:(1)转换为关节空间轨迹并按照第11.4.2节进行控制,或(2)在任务空间中表达机器人动力学和控制律。

第一个选项是将轨迹转换为关节空间。正运动学是$X = T(\theta)$和$V_b = J_b(\theta)\dot{\theta}$。然后使用逆运动学(第6章)从任务空间轨迹获得关节空间轨迹:
\begin{align}
\theta(t) &= T^{-1}(X(t)), \\
\dot{\theta}(t) &= J_b^{\dagger}(\theta(t))V_b(t), \\
\ddot{\theta}(t) &= J_b^{\dagger}(\theta(t))\left(\dot{V}_b(t) - \dot{J}_b(\theta(t))\dot{\theta}(t)\right).
\end{align}

这种方法的一个缺点是我们必须计算逆运动学、$J_b^{\dagger}$和$\dot{J}_b$,这可能需要大量的计算能力。

第二个选项是在任务空间坐标中表达机器人的动力学,如第8.6节所述。回顾任务空间动力学
\begin{equation}
F_b = \Lambda(\theta)\dot{V}_b + \eta(\theta, V_b).
\end{equation}

关节力和力矩$\tau$与在末端执行器坐标系中表示的wrench $F_b$通过$\tau = J_b^T(\theta)F_b$相关。

我们现在可以编写一个受关节坐标中计算力矩控制律(11.37)启发的任务空间控制律:
\begin{equation}
\tau = J_b^T(\theta)\left(\tilde{\Lambda}(\theta)\frac{d}{dt}([Ad_{X^{-1}X_d}]V_d) + K_p X_e + K_i\int_0^t X_e(t)dt + K_d V_e\right) + \tilde{\eta}(\theta, V_b),
\end{equation}
其中$\{\tilde{\Lambda}, \tilde{\eta}\}$表示控制器的动力学模型,$\frac{d}{dt}([Ad_{X^{-1}X_d}]V_d)$是在实际末端执行器坐标系$X$中表示的前馈加速度(这个项可以在接近参考状态的状态下近似为$\dot{V}_d$)。配置误差$X_e$满足$[X_e] = \log(X^{-1}X_d)$:$X_e$是在末端执行器坐标系中表示的twist,如果跟随单位时间,会将当前配置$X$移动到期望配置$X_d$。速度误差计算为
\begin{equation}
V_e = [Ad_{X^{-1}X_d}]V_d - V.
\end{equation}

变换$[Ad_{X^{-1}X_d}]$将在坐标系$X_d$中表示的参考twist $V_d$表示为在坐标系$X$中的末端执行器坐标系中的twist,其中实际速度$V$被表示,因此可以差分这两个表达式。

\section{力控制}

当任务不是在末端执行器处产生运动,而是向环境施加力和力矩时,需要力控制。纯力控制只有在环境在每个方向都提供阻力的情况下才可能(例如,如果末端执行器嵌入混凝土中或连接到在每个运动方向都提供阻力的弹簧)。纯力控制是一种抽象,因为机器人通常至少能够在某些方向上自由移动。然而,这是一个有用的抽象,它导致了第11.6节讨论的混合运动-力控制。

在理想力控制中,末端执行器施加的力不受施加到末端执行器的干扰运动的影响。这与理想运动控制的情况是对偶的,在理想运动控制中,运动不受干扰力的影响。

设$F_{\text{tip}}$是机械臂施加到环境的wrench。机械臂动力学可以写成
\begin{equation}
M(\theta)\ddot{\theta} + c(\theta, \dot{\theta}) + g(\theta) + b(\dot{\theta}) + J^T(\theta)F_{\text{tip}} = \tau,
\end{equation}
其中$F_{\text{tip}}$和$J(\theta)$定义在同一坐标系中(空间坐标系或末端执行器坐标系)。由于机器人在力控制任务期间通常移动缓慢(或根本不移动),我们可以忽略加速度和速度项,得到
\begin{equation}
g(\theta) + J^T(\theta)F_{\text{tip}} = \tau.
\end{equation}

在没有任何机器人末端执行器处力-力矩直接测量的情况下,可以单独使用关节角度反馈来实现力控制律
\begin{equation}
\tau = \tilde{g}(\theta) + J^T(\theta)F_d,
\end{equation}
其中$\tilde{g}(\theta)$是重力力矩的模型,$F_d$是期望的wrench。这个控制律需要良好的重力补偿模型以及对机器人关节产生的力矩的精确控制。

另一个解决方案是在机械臂和末端执行器之间为机器人臂配备六轴力-力矩传感器,以直接测量末端执行器wrench $F_{\text{tip}}$(图11.21)。考虑一个带前馈项和重力补偿的PI力控制器:
\begin{equation}
\tau = \tilde{g}(\theta) + J^T(\theta)\left(F_d + K_{fp}F_e + K_{fi}\int_0^t F_e(t)dt\right),
\end{equation}
其中$F_e = F_d - F_{\text{tip}}$,$K_{fp}$和$K_{fi}$分别是正定比例和积分增益矩阵。在完美重力建模的情况下,将力控制器(11.51)代入动力学(11.49),我们得到误差动力学
\begin{equation}
K_{fp}F_e + K_{fi}\int_0^t F_e(t)dt = 0.
\end{equation}

在(11.52)右边存在非零但恒定的力干扰的情况下(例如,由于$\tilde{g}(\theta)$的不正确模型),我们求导得到
\begin{equation}
K_{fp}\dot{F}_e + K_{fi}F_e = 0,
\end{equation}
表明对于正定$K_{fp}$和$K_{fi}$,$F_e$收敛到零。

控制律(11.51)简单且吸引人,但如果应用不当可能很危险。如果机器人没有东西可以推,它将在试图创建末端执行器力的失败尝试中加速。由于典型的力控制任务需要很少的运动,我们可以通过添加速度阻尼来限制这种加速度。这给出了修改的控制律
\begin{equation}
\tau = \tilde{g}(\theta) + J^T(\theta)\left(F_d + K_{fp}F_e + K_{fi}\int_0^t F_e(t)dt\right) - K_{\text{damp}}V,
\end{equation}
其中$K_{\text{damp}}$是正定的。

\section{混合运动-力控制}

大多数需要施加受控力的任务也需要产生受控运动。混合运动-力控制用于实现这一点。如果任务空间是$n$维的,那么我们在任何时间$t$都可以自由指定$2n$个力和运动中的$n$个;其他$n$个由环境决定。除了这个约束,我们也不应该在"相同方向"上指定力和运动,因为那样它们不是独立的。

\subsection{自然约束和人工约束}

一个特别有趣的情况发生在环境在$k$个方向上无限刚性(刚性约束)而在$n-k$个方向上无约束时。在这种情况下,我们不能选择指定$2n$个运动和力中的哪些——与环境的接触选择机器人可以自由施加力的$k$个方向和自由运动的$n-k$个方向。例如,考虑一个任务空间,其$n = 6$维为SE(3)。那么一个牢固抓住柜门的机器人有$6-k = 1$个末端执行器运动自由度,即绕柜铰链的旋转,因此有$k = 5$个力自由度;机器人可以施加任何绕铰链轴力矩为零的wrench而不移动门。

作为另一个例子,在黑板上写字的机器人可以自由控制进入板的力($k = 1$),但它不能穿透板;它可以自由移动$6-k = 5$个自由度(两个指定粉笔尖端在板平面内的运动,三个描述粉笔的方向),但它不能独立控制这些方向上的力。

作为最后一个例子,考虑一个机器人使用建模为刚性块的橡皮擦擦除无摩擦黑板(图11.22)。设$X(t) \in SE(3)$是块的坐标系$\{b\}$相对于空间坐标系$\{s\}$的配置。体坐标系twist和wrench分别写成$V_b = (\omega_x, \omega_y, \omega_z, v_x, v_y, v_z)$和$F_b = (m_x, m_y, m_z, f_x, f_y, f_z)$。保持与板的接触对twist施加$k = 3$个约束:
\begin{align}
\omega_x &= 0, \\
\omega_y &= 0, \\
v_z &= 0.
\end{align}

这些约束称为\textbf{自然约束},由环境指定。还有$6-k = 3$个wrench的自然约束:$m_z = f_x = f_y = 0$。根据自然约束,我们可以自由指定满足$k = 3$个速度约束的任何橡皮擦twist和满足$6-k = 3$个wrench约束的任何wrench(前提是$f_z < 0$,以保持与板的接触)。这些运动和力规范称为\textbf{人工约束}。下面是一个人工约束集合的例子,以及相应的自然约束:

\begin{center}
\begin{tabular}{|c|c|}
\hline
\textbf{自然约束} & \textbf{人工约束} \\
\hline
$\omega_x = 0$ & $m_x = 0$ \\
$\omega_y = 0$ & $m_y = 0$ \\
$m_z = 0$ & $\omega_z = 0$ \\
$f_x = 0$ & $v_x = k_1$ \\
$f_y = 0$ & $v_y = 0$ \\
$v_z = 0$ & $f_z = k_2 < 0$ \\
\hline
\end{tabular}
\end{center}

人工约束使橡皮擦以$v_x = k_1$移动,同时对板施加恒定力$k_2$。

\subsection{混合运动-力控制器}

我们现在回到设计混合运动-力控制器的问题。如果环境是刚性的,那么我们可以将任务空间中速度的$k$个自然约束表示为Pfaffian约束
\begin{equation}
A(\theta)V = 0,
\end{equation}
其中对于twist $V \in \mathbb{R}^6$,$A(\theta) \in \mathbb{R}^{k \times 6}$。这个公式包括完整约束和非完整约束。

如果机器人的任务空间动力学(第8.6节),在没有约束的情况下,由下式给出
\begin{equation}
F = \Lambda(\theta)\dot{V} + \eta(\theta, V),
\end{equation}
其中$\tau = J^T(\theta)F$是由执行器产生的关节力矩和力,那么根据第8.7节,约束动力学为
\begin{equation}
F = \Lambda(\theta)\dot{V} + \eta(\theta, V) + A^T(\theta)\lambda,
\end{equation}
其中$\lambda \in \mathbb{R}^k$是拉格朗日乘数,$F_{\text{tip}} = A^T(\theta)\lambda$是机器人施加到约束上的wrench。请求的wrench $F_d$必须位于$A^T(\theta)$的列空间中。

由于方程(11.55)必须在所有时间都满足,我们可以用时间导数替换它
\begin{equation}
A(\theta)\dot{V} + \dot{A}(\theta)V = 0.
\end{equation}

为了确保当系统状态已经满足$A(\theta)V = 0$时满足方程(11.57),任何请求的加速度$\dot{V}_d$应该满足$A(\theta)\dot{V}_d = 0$。

现在求解方程(11.56)得到$\dot{V}$,将结果代入(11.57),并求解$\lambda$,我们得到
\begin{equation}
\lambda = (A\Lambda^{-1}A^T)^{-1}(A\Lambda^{-1}(F - \eta) - A\dot{V}),
\end{equation}
其中我们使用了从方程(11.57)得到的$-A\dot{V} = \dot{A}V$。使用方程(11.58),我们可以计算机器人施加到约束上的wrench $F_{\text{tip}} = A^T(\theta)\lambda$。

将方程(11.58)代入方程(11.56)并操作,约束动力学(11.56)的$n$个方程可以表示为$n-k$个独立运动方程
\begin{equation}
P(\theta)F = P(\theta)(\Lambda(\theta)\dot{V} + \eta(\theta, V)),
\end{equation}
其中
\begin{equation}
P = I - A^T(A\Lambda^{-1}A^T)^{-1}A\Lambda^{-1}
\end{equation}
且$I$是单位矩阵。$n \times n$矩阵$P(\theta)$的秩为$n-k$,并将任意机械臂wrench $F$投影到使末端执行器沿约束切线移动的wrench子空间。秩$k$矩阵$I - P(\theta)$将任意wrench $F$投影到作用于约束的wrench子空间。因此$P$将$n$维力空间划分为处理运动控制任务的wrench和处理力控制任务的wrench。

我们的混合运动-力控制器简单地是任务空间运动控制器(从计算力矩控制律(11.47)导出)和任务空间力控制器(11.51)的总和,每个都投影到其适当的子空间以产生力。假设wrench和twist在末端执行器坐标系$\{b\}$中表示:
\begin{equation}
\begin{split}
\tau = J_b^T(\theta)\Bigg[&P(\theta)\left(\tilde{\Lambda}(\theta)\frac{d}{dt}([Ad_{X^{-1}X_d}]V_d) + K_p X_e + K_i\int_0^t X_e(t)dt + K_d V_e\right) \\
&+ (I - P(\theta))\left(F_d + K_{fp}F_e + K_{fi}\int_0^t F_e(t)dt\right) + \tilde{\eta}(\theta, V_b)\Bigg],
\end{split}
\end{equation}
其中第一项是运动控制,第二项是力控制,第三项是科里奥利和重力补偿。

由于两个控制器的动力学通过正交投影$P$和$I-P$解耦,控制器继承了各自子空间上单个力和运动控制器的误差动力学和稳定性分析。

在刚性环境中实现混合控制律(11.61)的一个困难是知道在任何时间活动的约束$A(\theta)V = 0$的形式。这对于指定期望的运动和力以及计算投影是必要的,但环境的任何模型都会有一些不确定性。处理这个问题的一种方法是使用实时估计算法基于力反馈识别约束方向。另一种是通过选择低反馈增益来牺牲一些性能,这使得运动控制器"软"且力控制器对力误差更容忍。我们还可以在机器人本身的结构中构建被动柔顺性以实现类似的效果。无论如何,由于关节和连杆中的柔顺性,一些被动柔顺性是不可避免的。

\section{阻抗控制}

刚性环境中的理想混合运动-力控制需要机器人阻抗的极端值,它表征端点运动作为干扰力函数的变化。理想运动控制对应于高阻抗(由于力干扰引起的运动变化很小),而理想力控制对应于低阻抗(由于运动干扰引起的力变化很小)。在实践中,机器人可实现的阻抗范围存在限制。

在本节中,我们考虑阻抗控制问题,其中要求机器人末端执行器呈现特定的质量、弹簧和阻尼特性。例如,用作触觉手术模拟器的机器人可以被要求模拟虚拟手术器械与虚拟组织接触时的质量、刚度和阻尼特性。

一个单自由度机器人呈现阻抗的动力学可以写成
\begin{equation}
m\ddot{x} + b\dot{x} + kx = f,
\end{equation}
其中$x$是位置,$m$是质量,$b$是阻尼,$k$是刚度,$f$是用户施加的力(图11.23)。粗略地说,如果$\{m, b, k\}$参数中的一个或多个(通常包括$b$或$k$)很大,我们说机器人呈现高阻抗。类似地,如果所有这些参数都很小,我们说阻抗很低。

更正式地,对方程(11.62)取拉普拉斯变换,我们得到
\begin{equation}
(ms^2 + bs + k)X(s) = F(s),
\end{equation}
阻抗由从位置扰动到力的传递函数定义,$Z(s) = F(s)/X(s)$。因此阻抗是频率相关的,低频响应由弹簧主导,高频响应由质量主导。导纳$Y(s)$是阻抗的逆:$Y(s) = Z^{-1}(s) = X(s)/F(s)$。

一个好的运动控制器的特点是高阻抗(低导纳),因为$\Delta X = Y \Delta F$。如果导纳$Y$很小,则力扰动$\Delta F$仅产生小的位置扰动$\Delta X$。类似地,一个好的力控制器的特点是低阻抗(高导纳),因为$\Delta F = Z \Delta X$,小的$Z$意味着运动扰动仅产生小的力扰动。

阻抗控制的目标是实现任务空间行为
\begin{equation}
M\ddot{x} + B\dot{x} + Kx = f_{\text{ext}},
\end{equation}
其中$x \in \mathbb{R}^n$是最小坐标集中的任务空间配置,例如$x \in \mathbb{R}^3$;$M$、$B$和$K$是机器人要模拟的正定虚拟质量、阻尼和刚度矩阵,$f_{\text{ext}}$是施加到机器人的力,可能由用户施加。$M$、$B$和$K$的值可能会改变,取决于虚拟环境中的位置,例如为了表示不同的物体,但我们专注于恒定值的情况。

有两种常见的方法来实现行为(11.64):
\begin{itemize}
\item 机器人感测端点运动$x(t)$并指令关节力矩和力以创建$-f_{\text{ext}}$,即向用户显示的力。这样的机器人称为\textbf{阻抗控制},因为它实现了从运动到力的传递函数$Z(s)$。理论上,阻抗控制的机器人应该只与导纳型环境耦合。
\item 机器人使用手腕力-力矩传感器感测$f_{\text{ext}}$并控制其运动作为响应。这样的机器人称为\textbf{导纳控制},因为它实现了从力到运动的传递函数$Y(s)$。理论上,导纳控制的机器人应该只与阻抗型环境耦合。
\end{itemize}

\subsection{阻抗控制算法}

在阻抗控制算法中,使用编码器、转速计和可能的加速度计来估计关节和端点位置、速度和可能的加速度。通常,阻抗控制的机器人不配备手腕力-力矩传感器,而是依靠其精确控制关节力矩的能力来呈现适当的末端执行器力$-f_{\text{ext}}$(方程(11.64))。一个好的控制律可能是
\begin{equation}
\tau = J^T(\theta)\left(\tilde{\Lambda}(\theta)\ddot{x} + \tilde{\eta}(\theta, \dot{x}) - (M\ddot{x} + B\dot{x} + Kx)\right),
\end{equation}
其中任务空间动力学模型$\{\tilde{\Lambda}, \tilde{\eta}\}$用坐标$x$表示。添加末端执行器力-力矩传感器允许使用反馈项来更接近地实现期望的交互力$-f_{\text{ext}}$。

在控制律(11.65)中,假设$\ddot{x}$、$\dot{x}$和$x$是直接测量的。加速度$\ddot{x}$的测量可能是噪声的,并且存在在感测到加速度后试图补偿机器人质量的问题。因此,消除质量补偿项$\tilde{\Lambda}(\theta)\ddot{x}$并设置$M = 0$并不罕见。臂的质量对用户来说是明显的,但阻抗控制的机械臂通常设计为轻量级。假设小速度并用更简单的重力补偿模型替换非线性动力学补偿也不罕见。

当(11.65)用于模拟刚性环境(大$K$的情况)时,可能会出现问题。一方面,位置的微小变化(例如由编码器测量)导致电机力矩的大变化。这种有效的高增益,加上延迟、传感器量化和传感器误差,可能导致振荡行为或不稳定。另一方面,在模拟低阻抗环境时,有效增益很低。轻量级可反向驱动的机械臂可以很好地模拟这种环境。

\subsection{导纳控制算法}

在导纳控制算法中,用户施加的力$f_{\text{ext}}$由手腕负载单元感测,机器人响应满足方程(11.64)的末端执行器加速度。一种简单的方法是根据下式计算期望的末端执行器加速度
\begin{equation}
M\ddot{x}_d + B\dot{x} + Kx = f_{\text{ext}},
\end{equation}
其中$(x, \dot{x})$是当前状态。求解,我们得到
\begin{equation}
\ddot{x}_d = M^{-1}(f_{\text{ext}} - B\dot{x} - Kx).
\end{equation}

对于由$x = J(\theta)\dot{\theta}$定义的雅可比$J(\theta)$,期望的关节加速度$\ddot{\theta}_d$可以求解为
\begin{equation}
\ddot{\theta}_d = J^{\dagger}(\theta)(\ddot{x}_d - \dot{J}(\theta)\dot{\theta}),
\end{equation}
并使用逆动力学来计算指令的关节力和力矩$\tau$。当目标是仅模拟弹簧或阻尼器时,可以获得这个控制律的简化版本。为了在面对噪声力测量时使响应更平滑,可以对力读数进行低通滤波。

模拟低质量环境对于导纳控制的机器人来说是具有挑战性的,因为小的力产生大的加速度。有效的大增益可能产生不稳定。然而,高度齿轮传动的机器人的导纳控制可以很好地模拟刚性环境。

\section{低级关节力/力矩控制}

在本章中,我们一直假设每个关节产生请求的力矩或力。在实践中,这个理想并没有完全实现,有不同的方法来近似它。下面列出了一些使用电机(第8.9.1节)的最常见方法,以及它们相对于先前列出方法的优缺点。这里我们假设一个旋转关节和一个旋转电机。

\textbf{直接驱动电机的电流控制} 在这种配置中,每个关节有一个电机放大器和没有齿轮箱的电机。电机的力矩近似遵循关系$\tau = k_t I$,即力矩与通过电机的电流成正比。放大器接受请求的力矩,除以力矩常数$k_t$,并产生电机电流$I$。为了创建期望的电流,与放大器集成的电流传感器连续测量通过电机的实际电流,放大器使用局部反馈控制回路来调整电机上的时间平均电压以实现期望的电流。这个局部反馈回路以比生成请求力矩的控制回路更高的速率运行。一个典型的例子是局部电流控制回路为10 kHz,请求关节力矩的外层控制回路为1 kHz。

这种配置的一个问题是,通常无齿轮的电机必须相当大才能为应用创建足够的力矩。如果电机固定在地面上并通过电缆或闭链连杆连接到末端执行器,这种配置可以工作。如果电机在移动,例如串联链关节处的电机,大型无齿轮电机通常不实用。

\textbf{齿轮电机的电流控制} 这种配置与之前的配置类似,除了电机有一个齿轮箱(第8.9.1节)。齿轮比$G > 1$增加了关节可用的力矩。

\textbf{优点:}较小的电机可以提供必要的力矩。电机还在更高的速度下运行,在那里它将电能转换为机械能的效率更高。

\textbf{缺点:}齿轮箱引入了间隙(齿轮箱的输出可以在输入不移动的情况下移动,使得接近零速度的运动控制具有挑战性)和摩擦。通过使用特定类型的齿轮,如谐波驱动齿轮,可以几乎消除间隙。然而,摩擦无法消除。齿轮箱输出的标称力矩是$Gk_t I$,但齿轮箱中的摩擦减少了可用力矩并在实际产生的力矩中产生显著的不确定性。

\textbf{带局部应变计反馈的齿轮电机电流控制} 这种配置与之前的配置类似,除了谐波驱动齿轮配备应变计,感测在齿轮箱输出处实际传递的力矩量。这个力矩信息由放大器在局部反馈控制器中使用,以调整电机中的电流,从而实现请求的力矩。

\textbf{优点:}将传感器放在齿轮箱的输出处允许补偿摩擦不确定性。

\textbf{缺点:}关节配置有额外的复杂性。此外,谐波驱动齿轮通过在齿轮组中引入一些扭转柔顺性来实现接近零间隙,并且由于存在这个扭转弹簧而增加的动力学可能使高速运动控制复杂化。

\textbf{串联弹性执行器} 串联弹性执行器(SEA)由带齿轮箱(通常是谐波驱动齿轮箱)的电机和将齿轮箱输出连接到执行器输出的扭转弹簧组成。它与之前的配置类似,除了添加的弹簧的扭转弹簧常数远低于谐波驱动齿轮的弹簧常数。弹簧的角偏转$\Delta\phi$通常由光学、磁性或电容编码器测量。传递到执行器输出的力矩是$k\Delta\phi$,其中$k$是扭转弹簧常数。弹簧的偏转被馈送到局部反馈控制器,该控制器控制电机的电流以实现期望的弹簧偏转,从而实现期望的力矩。

\textbf{优点:}添加扭转弹簧使关节自然"柔软",因此非常适合人机交互任务。它还保护齿轮箱和电机免受输出处的冲击,例如当输出连杆撞击环境中的硬物时。

\textbf{缺点:}关节配置有额外的复杂性。此外,由于较软的弹簧而增加的动力学使得控制输出处的高速或高频运动更具挑战性。

在2011年,NASA的Robonaut 2(R2)成为太空中的第一个人形机器人,在国际空间站上执行操作。Robonaut 2包含许多SEA,包括图11.24中显示的髋关节执行器。

\section{其他主题}

\textbf{鲁棒控制} 虽然所有稳定的反馈控制器都对不确定性赋予一定程度的操作鲁棒性,但鲁棒控制领域涉及设计明确保证机器人性能的控制器,该机器人受到有界参数不确定性的影响,例如其惯性特性中的不确定性。

\textbf{自适应控制} 机器人的自适应控制涉及在执行期间估计机器人的惯性或其他参数,并实时更新控制律以纳入这些估计。

\textbf{迭代学习控制} 迭代学习控制(ILC)通常专注于重复任务。如果机器人一遍又一遍地执行相同的取放操作,则来自先前执行的轨迹误差可用于修改下一次执行的前馈控制。通过这种方式,机器人随着时间的推移改善其性能,将执行误差驱动到零。这种类型的学习控制与自适应控制的不同之处在于,"学习"的信息通常是非参数的,并且仅对单个轨迹有用。然而,ILC可以解释在特定模型中未参数化的影响。

\section{总结}

\begin{itemize}
\item 机器人控制器将任务规范转换为执行器处的力和力矩。
\item 控制策略包括运动控制、力控制、混合运动-力控制和阻抗控制。
\item 反馈控制使用位置、速度和力传感器来测量机器人的实际行为,并将其与期望行为进行比较。
\item 误差动力学描述了关节误差的演化,控制器的目的是使误差趋于零。
\item 线性误差动力学可以用线性常微分方程描述,稳定性由特征方程的根决定。
\item 一阶和二阶误差动力学提供了对控制器行为的良好理解。
\item PD控制是最基本的反馈控制方法,但存在稳态误差。
\item 重力补偿可以消除重力引起的稳态误差。
\item PID控制通过积分项可以消除持续干扰引起的稳态误差。
\item 前馈控制可以改善跟踪性能,通过补偿期望轨迹的动力学。
\item 计算力矩控制结合前馈和反馈,提供高精度的轨迹跟踪。
\item 力控制用于机器人与环境接触的情况。
\item 混合运动-力控制在不同的任务空间方向上分别控制运动和力。
\item 阻抗控制使机器人表现得像具有特定机械阻抗的系统。
\end{itemize}

\section{练习}

\begin{enumerate}
\item 将以下机器人任务分类为运动控制、力控制、混合运动-力控制、阻抗控制或某些组合。证明你的答案。
   \begin{itemize}
   \item[(a)] 用螺丝刀拧紧螺丝。
   \item[(b)] 沿地板推箱子。
   \item[(c)] 倒一杯水。
   \item[(d)] 与人类握手。
   \item[(e)] 投掷棒球击中目标。
   \item[(f)] 铲雪。
   \item[(g)] 挖洞。
   \item[(h)] 背部按摩。
   \item[(i)] 用吸尘器清洁地板。
   \item[(j)] 端着一盘玻璃杯。
   \end{itemize}

\item 欠阻尼二阶系统的2\%稳定时间约为$t = 4/(\zeta\omega_n)$,对于$e^{-\zeta\omega_n t} = 0.02$。5\%稳定时间是多少?

\item 求解任何常数并给出具有$\omega_n = 4$、$\zeta = 0.2$、$\theta_e(0) = 1$和$\dot{\theta}_e(0) = 0$的欠阻尼二阶系统的特定方程。计算阻尼自然频率、近似超调量和2\%稳定时间。在计算机上绘制解并测量精确的超调量和稳定时间。

\item 求解任何常数并给出具有$\omega_n = 10$、$\zeta = 0.1$、$\theta_e(0) = 0$和$\dot{\theta}_e(0) = 1$的欠阻尼二阶系统的特定方程。计算阻尼自然频率。在计算机上绘制解。

\item 考虑重力场中的摆,$g = 10$ m/s$^2$。摆由1 m无质量杆末端的2 kg质量组成。摆关节的粘性摩擦系数为$b = 0.1$ N·m·s/rad。
   \begin{itemize}
   \item[(a)] 用$\theta$写出摆的运动方程,其中$\theta = 0$对应于"下垂"配置。
   \item[(b)] 关于稳定的"下垂"平衡点线性化运动方程。为此,用泰勒展开中的线性项替换$\theta$中的任何三角函数项。给出线性化动力学$m\ddot{\theta} + b\dot{\theta} + k\theta = 0$中的有效质量和弹簧常数$m$和$k$。在稳定平衡点,阻尼比是多少?系统是欠阻尼、临界阻尼还是过阻尼?如果是欠阻尼,阻尼自然频率是多少?收敛到平衡的时间常数和2\%稳定时间是多少?
   \item[(c)] 现在写出$\theta = 0$在平衡直立配置时的线性化运动方程。有效弹簧常数$k$是多少?
   \item[(d)] 你在摆的关节处添加电机以稳定直立位置,并选择P控制器$\tau = K_p\theta$。$K_p$的什么值使直立位置稳定?
   \end{itemize}

\item 为形式为$m\ddot{x} + b\dot{x} + kx = f$的单自由度质量-弹簧-阻尼器系统开发控制器,其中$f$是控制力,$m = 4$ kg,$b = 2$ N·s/m,$k = 0.1$ N/m。
   \begin{itemize}
   \item[(a)] 未控制系统的阻尼比是多少?未控制系统是过阻尼、欠阻尼还是临界阻尼?如果是欠阻尼,阻尼自然频率是多少?收敛到原点的时间常数是多少?
   \item[(b)] 选择P控制器$f = K_p x_e$,其中$x_e = x_d - x$是位置误差,$x_d = 0$。$K_p$的什么值产生临界阻尼?
   \item[(c)] 选择D控制器$f = K_d \dot{x}_e$,其中$\dot{x}_d = 0$。$K_d$的什么值产生临界阻尼?
   \item[(d)] 选择产生临界阻尼和0.01 s的2\%稳定时间的PD控制器。
   \item[(e)] 对于上述PD控制器,如果$x_d = 1$且$\dot{x}_d = \ddot{x}_d = 0$,当$t$趋于无穷时稳态误差$x_e(t)$是多少?稳态控制力是多少?
   \item[(f)] 现在为$f$插入PID控制器。假设$x_d \neq 0$且$\dot{x}_d = \ddot{x}_d = 0$。用$x_e$、$\dot{x}_e$、$x_e$和$\int x_e(t)dt$在左边和常数强迫项在右边写出误差动力学。(提示:你可以将$kx$写成$-k(x_d - x) + kx_d$。)对这个方程求时间导数,并给出$K_p$、$K_i$和$K_d$的稳定性条件。证明使用PID控制器可以实现零稳态误差。
   \end{itemize}

\item 单自由度机器人和机器人控制器的仿真。
   \begin{itemize}
   \item[(a)] 使用第11.4.1节给出的模型参数,编写一个单关节机器人的仿真器,该机器人由在重力作用下旋转连杆的电机组成。仿真器应包括:(1)一个动力学函数,以机器人的当前状态和电机施加的力矩作为输入,并给出机器人的加速度作为输出;以及(2)一个数值积分器,使用动力学函数在一系列时间步长$\Delta t$上计算系统的新状态。一阶欧拉积分方法足以解决这个问题(例如,$\theta(k+1) = \theta(k) + \dot{\theta}(k)\Delta t$,$\dot{\theta}(k+1) = \dot{\theta}(k) + \ddot{\theta}(k)\Delta t$)。用两种方式测试仿真器:(1)机器人在$\theta = -\pi/2$处静止开始,施加0.5 N·m的恒定力矩;以及(2)机器人在$\theta = -\pi/4$处静止开始,施加零力矩。对于两个例子,绘制足够长时间的位置作为时间的函数以看到基本行为。确保行为合理。$\Delta t$的合理选择是1 ms。
   \item[(b)] 向仿真器添加两个函数:(1)一个轨迹生成器函数,接受当前时间并返回机器人的期望状态和加速度;以及(2)一个控制函数,接受机器人的当前状态和来自轨迹生成器的信息,并返回控制力矩。最简单的轨迹生成器将返回$\theta = \theta_{d1}$和$\dot{\theta} = \ddot{\theta} = 0$对于所有时间$t < T$,以及$\theta = \theta_{d2} \neq \theta_{d1}$和$\dot{\theta} = \ddot{\theta} = 0$对于所有时间$t \geq T$。这个轨迹是位置的阶跃函数。对控制函数使用PD反馈控制器,并设置$K_p = 10$ N·m/rad。对于良好调整的$K_d$选择,给出$K_d$(包括单位)并在2秒内绘制位置作为时间的函数,初始状态在$\theta = -\pi/2$处静止,阶跃轨迹为$\theta_{d1} = -\pi/2$和$\theta_{d2} = 0$。阶跃发生在$T = 1$ s。
   \item[(c)] 演示产生(1)超调和(2)无超调的缓慢响应的两种不同的$K_d$选择。给出增益和位置图。
   \item[(d)] 向原始良好调整的PD控制器添加非零$K_i$以消除稳态误差。给出PID增益并绘制阶跃测试的结果。
   \end{itemize}

\item 修改练习11.7中的单关节机器人仿真,以模拟从电机到连杆的柔性传动,刚度为500 N·m/rad。调整PID控制器以在期望轨迹从$\theta = -\pi/2$到$\theta = 0$的阶跃过渡时给出良好响应。给出增益并绘制响应。

\item 两自由度机器人和机器人控制器的仿真(图11.25)。
   \begin{itemize}
   \item[(a)] 动力学。推导重力作用下2R机器人的动力学(图11.25)。连杆$i$的质量为$m_i$,质心距离关节$r_i$,连杆$i$关于关节的标量惯性为$I_i$,连杆$i$的长度为$L_i$。关节处没有摩擦。
   \item[(b)] 直接驱动。假设每个关节由无齿轮箱的DC电机直接驱动。每个电机都带有定子的质量$m_i^{\text{stator}}$和惯性$I_i^{\text{stator}}$以及转子的质量$m_i^{\text{rotor}}$和惯性$I_i^{\text{rotor}}$的规格。对于关节$i$处的电机,定子连接到连杆$i-1$,转子连接到连杆$i$。连杆是质量为$m_i$、长度为$L_i$的薄均匀密度杆。根据上面给出的量,对于每个连杆$i \in \{1, 2\}$,给出关于关节的总惯性$I_i$、质量$m_i$和从关节到质心的距离$r_i$的方程。考虑如何将电机的质量和惯性分配给不同的连杆。
   \item[(c)] 齿轮机器人。假设电机$i$具有齿轮比$G_i$的齿轮箱,并且齿轮箱本身是无质量的。如上面的(b)部分,对于每个连杆$i \in \{1, 2\}$,给出关于关节的总惯性$I_i$、质量$m_i$和从关节到质心的距离$r_i$的方程。
   \item[(d)] 仿真和控制。如练习11.7,编写一个具有(至少)四个函数的仿真器:动力学函数、数值积分器、轨迹生成器和控制器。假设关节处零摩擦,$g = 9.81$ m/s$^2$在指示方向,$L_i = 1$ m,$r_i = 0.5$ m,$m_1 = 3$ kg,$m_2 = 2$ kg,$I_1 = 2$ kg·m$^2$,$I_2 = 1$ kg·m$^2$。编程PID控制器,找到给出良好响应的增益,并绘制关节角度作为时间的函数,参考轨迹在$t < 1$ s时恒定在$(\theta_1, \theta_2) = (-\pi/2, 0)$,在$t \geq 1$ s时恒定在$(\theta_1, \theta_2) = (0, -\pi/2)$。机器人的初始状态在$(\theta_1, \theta_2) = (-\pi/2, 0)$处静止。
   \item[(e)] 力矩限制。真实电机对可用力矩有限制。虽然这些限制通常依赖于速度,但这里我们假设每个电机的力矩限制与速度无关,$\tau_i \leq |\tau_i^{\max}|$。假设$\tau_1^{\max} = 100$ N·m和$\tau_2^{\max} = 20$ N·m。控制律可能请求更大的力矩,但实际力矩在这些值处饱和。重新运行(d)中的PID控制仿真,并绘制力矩以及位置作为时间的函数。
   \item[(f)] 摩擦。向每个关节添加粘性摩擦系数$b_i = 1$ N·m·s/rad。重新运行仿真并比较结果。
   \end{itemize}

\item 考虑一个三阶误差动力学系统:
   \begin{equation}
   \dddot{\theta}_e + a_2\ddot{\theta}_e + a_1\dot{\theta}_e + a_0\theta_e = 0
   \end{equation}
   使用Routh-Hurwitz稳定性判据,确定使系统稳定的$a_0$、$a_1$和$a_2$的条件。

\item 对于任务空间控制,解释为什么使用$[Ad_{X^{-1}X_d}]V_d$而不是简单地使用$V_d$来表示期望速度。给出一个具体例子说明差异。

\item 考虑混合运动-力控制。如果约束矩阵$A(\theta)$不准确,会对控制性能产生什么影响?提出一种方法来处理这种不确定性。

\item 分析阻抗控制和导纳控制的区别。在什么情况下应该使用阻抗控制而不是导纳控制?给出一个具体应用例子。
\end{enumerate}

\end{document}

